\chapter{Estudo de Caso}

Este capítulo apresenta o estudo de caso conduzido com professores de \textit{introdução à programação} com o objetivo de investigar a viabilidade, utilidade prática e os desafios associados ao uso de templates multicamadas proposto neste trabalho. A pesquisa adota uma abordagem integrada à pesquisa-ação, com uso de métodos mistos (quantitativos e qualitativos). O estudo foi conduzido e estruturado para oferecer evidências que respondem às três questões de pesquisa (QP1, QP2 e QP3) conforme as diretrizes metodológicas recomendadas pela \gls{ceie} para estudos de caso.

\section{Objetivo do Estudo}
O objetivo deste estudo de caso é investigar a aplicabilidade da geração automática de questões utilizando templates multicamadas no contexto do ensino de programação, com base na experiência de professores na construção desses templates, e em suas percepções sobre a utilidade e os desafios da abordagem. Para isso, são apresentados a abordagem utilizada,  os procedimentos envolvidos e as métricas adotadas, que fundamentam a coleta e análise dos dados qualitativos e quantitativos que serão discutidos nas seções seguintes.
Embora não tenha sido possível aplicar o feedback automatizado aos participantes devido ao tempo limitado para o desenvolvimento deste estudo, a literatura  indica que este recurso tem potencial significativo para contribuir no aprendizado, como evidenciado por trabalhos como o de \parencite{vanpraet2024} e \parencite{fung2024} indicam que a incorporação de feedback automatizado se utilizado corretamente pode aumentar de forma significativa o desempenho dos estudantes.

\section{Metodologia do Estudo de Caso}

Este estudo adotou uma abordagem mista que combina técnicas quantitativas e qualitativas dentro de um ciclo de pesquisa. A opção por integrar essas abordagens decorreu de duas necessidades complementares: mensurar, de forma objetiva, o tempo gasto, o número de questões produzidas e a taxa de reaproveitamento dos \textit{templates}, ao mesmo tempo em que se investigavam as percepções, dificuldades e sugestões dos professores.


\subsection{Perfil dos Professores}
Para garantir que os resultados reflitam um panorama representativo dos professores de Introdução a programação, foi definido critérios de inclusão e coleta de informação de para cada professor com base nos seguintes aspectos: 

\begin{itemize}
    \item \textbf{Experiência recente}: lecionam ou lecionaram disciplinas introdutórias de programação em pelo menos 2 semestres nos cinco anos anteriores ao estudo.
    \item \textbf{Disponibilidade}: Compromisso para criar e avaliar templates e responder o questionário.
\end{itemize}

As Figuras \ref{fig:nivel-escolaridade}, \ref{fig:anos-docencia} e \ref{fig:linguagens-ensinadas} apresentam respectivamente, a distribuição dos participantes em relação ao nível de escolaridade, aos anos de experiência docente e às linguagens de programação ensinadas.

\begin{figure}[ht]
	\centering
	\includegraphics[width=16cm]{./imagens/capitulo8/1-escolaridade}
	\caption{Perfil Profissional 1 (Elaboração própria, 2025) }
	\label{fig:nivel-escolaridade}
\end{figure}
\begin{figure}[ht]
	\centering
	\includegraphics[width=16cm]{./imagens/capitulo8/2-anos-docencia}
	\caption{Perfil Profissional 2 (Elaboração própria, 2025) }
	\label{fig:anos-docencia}
\end{figure}
\begin{figure}[ht]
	\centering
	\includegraphics[width=16cm]{./imagens/capitulo8/2-linguagens-ensinadas}
	\caption{Perfil Profissional 3 (Elaboração própria, 2025) }
	\label{fig:linguagens-ensinadas}
\end{figure}

 
\subsection{Variáveis levantadas pelo questionário}
O formulário (ver Apêndice~\ref{apendice:formulario}) foi dividido em 4 blocos, cada um mapeando uma das questões de pesquisa:

\begin{table}[H]
    \centering
    \begin{tabular}{|p{4cm}|p{5.4cm}|p{6cm}|}
        \hline
        \textbf{Bloco} & \textbf{Variáveis coletadas} & \textbf{Objetivo} \\ \hline
        Perfil profissional & Titulação, anos de docência, linguagens mais ensinadas. & Relacionar experiência ao \textbf{tempo médio} de criação de templates. \\ \hline
        Prática com templates & Tempo gasto, número de questões geradas, taxa de reaproveitamento. & Medir \textbf{viabilidade} e \textbf{utilidade} operacional. \\ \hline
        Percepções e desafios & Dificuldades encontradas, contribuição percebida, preferências de uso. & Identificar \textbf{barreiras} técnicas e pedagógicas. \\ \hline
        Adaptação de questões existentes & Grau de facilidade em converter questões próprias para o formato multicamada. & Verificar \textbf{capacidade de transformação} sem suporte externo. \\ \hline
    \end{tabular}
    \caption{Estrutura do questionário (Elaboração própria, 2025)}
    \label{tab:questionario-objetivos}
\end{table}

\subsection*{Etapas do Estudo}
O processo metodológico foi dividido em cinco etapas:

\begin{enumerate}
    \item \textbf{Apresentação da proposta}: Foi exibido um vídeo curto dividido em três blocos (direcionamento, questões base e estrutura do modelo), com explicações sobre os fundamentos da geração automática de questões usando templates multicamadas, além de orientações de uso e exemplos práticos. O objetivo foi garantir a compreensão do assunto por parte dos professores.

    \item \textbf{Criação de templates pelos professores}: Após a apresentação do modelo, os professores elaboraram templates utilizando uma estrutura previamente definida. A orientação fornecida foi dividida em três tópicos principais:
    \begin{itemize}
        \item \textbf{Direcionamento}: Fundamentos e exemplos com perguntas-chave para ajudar na formulação do enunciado e identificação dos elementos variáveis.
        \item \textbf{Questões base}: Orientações para identificar os aspectos fundamentais do problema a ser transformado em questão.
        \item \textbf{Estrutura modelo}: Guia de construção do template, indicando como organizar o enunciado, variáveis e condições.
    \end{itemize}
    
    Os professores puderam utilizar ferramentas de apoio, como o \gls{chatgpt}, para auxiliar na criação e sugestão de variações e contextos. Esse suporte teve como objetivo facilitar a construção dos templates, ampliar as possibilidades de formulação e oferecer maior adaptabilidade.
\end{enumerate}


\section{Metodologia do Estudo de Caso}

Este estudo adotou uma abordagem mista que combina técnicas quantitativas e qualitativas dentro de um ciclo de pesquisa. A opção por integrar essas abordagens decorreu de duas necessidades complementares: mensurar, de forma objetiva, o tempo gasto, o número de questões produzidas e a taxa de reaproveitamento dos \textit{templates}, ao mesmo tempo em que se investigavam as percepções, dificuldades e sugestões dos professores, o processo foi divido em cinco etapas que são:  

\begin{enumerate}
    \item \textbf{Apresentação da proposta}:  Foi exibido um vídeo curto dividido em três blocos: (direcionamento, questões base e estrutura do modelo) com explicações sobre os fundamentos da geração automática de questões usando templates multicamadas, além de orientações de uso e exemplos práticos de aplicação. O objetivo foi garantir  compreensão do assunto por parte dos professores.
    
    \item \textbf{Criação de templates pelos professores}:  Os professores participantes após apresentação do modelo vão elaborar os templates utilizando uma estrutura previamente definida. A orientação fornecida foi divida em três tópicos principais:
    \begin{itemize}
        \item \textbf{Direcionamento} : fundamentos e exemplos com perguntas-chave para ajudar na formulação do enunciado e identificação dos elementos variáveis.
        \item \textbf{Questões base} : Orientações para identificar os aspectos fundamentais do problema que desejam transformar em uma questão.
        \item \textbf{Estrutura modelo} : Um guia de construção do template, indicando com organizar o template, variáveis e condições.
    \end{itemize}

Os professores puderam utilizar ferramentas de auxilio como o \gls{chatgpt}, para ajudar na criação e sugestão de variações e contextos. Esse suporte tem como objetivo facilitar a criação dos templates, ampliar as possibilidades de formulação e oferecer templates mais adaptáveis.

    \item \textbf{Objetivos do Estudo}:  O objetivo geral deste estudo é avaliar a aplicabilidade da utilização de templates multicamadas na geração automática de questões de programação, com base na experiência prática dos professores envolvidos no processo. Os objetivos específicos que orientam este estudo são:

       \begin{itemize}
        \item \textbf{Medir a viabilidade operacional} : Considerar  o esforço envolvido e nível de sucesso na criação das questões a partir dos templates.
        \item \textbf{Verificar a utilidade prática} : Analisar a economia de tempo, a diversidade das questões geradas e o nível de engajamento do professor.
        \item \textbf{Identificar dificuldades} : Verificar aspectos relacionados à curva de aprendizagem, à definição e ao uso adequado de variáveis, bem como as possíveis barreiras técnicas encontradas durante a construção dos templates. 
    \end{itemize}
   
    \item \textbf{Variáveis levantadas pelo questionário}:  O formulário (ver Apêndice 1) foi dividido em 4 blocos, cada um deles mapeando uma questão de pesquisa: 



\subsection{Métricas e indicadores}

Para avaliar a utilidade prática dos templates, foram definidos três indicadores. Cada um deles foi escolhido por sua relevância direta às questões de pesquisa e por oferecer evidências sobre o esforço exigido, a produtividade e o potencial de reutilização dos templates gerados.
    \begin{itemize}
        \item \textbf{Tempo médio pra criação de um template} : intervalo em minutos, entre o inicio e o momento em que o professor considera o template pronto pra uso. Valores baixos indicam maior viabilidade operacional.
        \item \textbf{Numero de questões geradas por template} : quantidade total de instâncias de questões produzidas a partir de um único template, considerando as combinações válidas de variáveis, quanto maior o repertório de questões geradas maior o poder de generalização do template.
        \item  \textbf{Sucesso em converter a questão existente em template }: indica se foi possível ou não transformar uma questão existente em um template funcional, mantendo sua essência pedagógica. 
    \end{itemize}

    
\end{enumerate}



\section{Resultados Observados}

O estudo revelou que os professores conseguiram elaborar templates funcionais e diversificados, utilizando as orientações fornecidas. No entanto, alguns desafios foram identificados:

\begin{enumerate}
    \item \textbf{Dificuldade Inicial}: Professores com menor familiaridade com conceitos de questões baseada em templates demonstraram dificuldades em identificar elementos variáveis e combinar os valores.
    \item \textbf{Uso da IA Gnerativa} : A integração do ChatGPT foi bem recebida, e os participantes relataram que as sugestões fornecidas pela ferramenta ajudaram na proposta de contextos diversificados.
    \item \textbf{Percepção Geral} : Os professores consideraram o modelo útil para reduzir o tempo de elaboração de questões, no entanto preferiram que os templates já estivessem prontos pra utilizar, e após construídos pudessem acrescentar ou editar a estrutura conforme a necessidade ao invés de criar do zero.
\end{enumerate}


\begin{table}
\centering

\begin{tabular}{l}
 \\

\end{tabular}

\end{table}

\begin{table}
\centering

\begin{tabular}{l}
\textbf{Resultados} \\

\end{tabular}

\end{table}

\begin{table}
\centering

\begin{tabular}{l}
Quantitativos (tabelas/gráficos) e qualitativos (depoimentos). \\

\end{tabular}

\end{table}

 
\section{Beneficios e Desafios Percebidos} 

redução do esfoço de elaboração, maior diversidade de exercicio

identificar corretamente as variaveis e seus valores
curva de aprendizagem para manipular o template
preferencia em usar templates-base prontos em vez de começar do zero.

\section{Considerações Finais do Estudo de Caso}
A aplicação foi realizada com um número reduzido de professores, o que limita a generalização dos resultados. Estudos futuros poderão expandir o alcance para incluir mais participantes para validar a proposta. Apesar das limitações, a abordagem de geração automática de questões demonstrou um potencial significativo para criar questões em escala e reduzir o esforço necessário para a construção de conteúdos por parte dos professores.


RESPOSTA AS QUESTÕES DE PESQUISA COM OS INDICADORES 
VIABILIDADE : INDICADOR DE TEMPO
UTILIDADE : TAXA DE APROVEITAMENTO 
DESAFIOS : FALTA DE MODELOS PRÉ-PRONTOS

AMOSTRA REDUZIDA LIMITA A GENERALIZAÇÃO


\subsubsection{\textbf{Resultados}}

\paragraph{\textbf{7.8.1 Eficiência do processo}}

Tempo médio para criar o primeiro template: 27 min (desvio-padrão 6,2 min). Média de 12,4 questões por template e reaproveitamento de 83 \%.

\paragraph{\textbf{7.8.2 Depoimentos dos professores}}

\begin{quote}
“Com o guia, consegui transformar rapidamente uma questão antiga em um template; a IA me ajudou a criar três variações de contexto em minutos.” — P2

\end{quote}

\paragraph{\textbf{7.8.3 Benefícios percebidos}}

Redução do esforço de elaboração, maior diversidade de exercícios e incremento no engajamento dos alunos.

\paragraph{\textbf{7.8.4 Desafios e dificuldades}}

\begin{itemize}
    \item Identificar corretamente as variáveis e seus valores.
    \item Curva de aprendizagem para manipular placeholders.
    \item Preferência por templates-base prontos em vez de começar do zero.
\end{itemize}
 

\paragraph{\textbf{Principais Desafios Identificados}}

\begin{itemize}
    \item \textbf{Curva de aprendizagem inicial} (especialmente na identificação de variáveis e condições);
    \item \textbf{Necessidade de modelos‐base} para acelerar a adoção;
    \item \textbf{Integração com sistemas de avaliação existentes} (Moodle, Google Forms, etc.), que exigirá conversores automáticos — apontado como trabalho futuro.
\end{itemize}
\textbf{QP3 – Desafios:} maior dificuldade em definir variáveis e preferências por templates‐base prontos; 60 \% solicitaram um repositório inicial de modelos. 