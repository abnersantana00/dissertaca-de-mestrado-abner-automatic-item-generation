\chapter{Estudo de Caso}

Esta seção apresenta o estudo de caso conduzido para avaliar a viabilidade, utilidade e relevância da proposta de geração automática de questões de programação baseada em templates multicamadas. O estudo foi estruturado para incluir a criação de templates por professores, coleta de feedback e uma análise sobre a compreensão e percepção dos participantes em relação à proposta. A seguir,  serão detalhadas as etapas do estudo e os principais resultados observados. 

\section{Objetivo do Estudo}
O objetivo principal deste estudo de caso foi validar a aplicabilidade da geração automática de questões usando templates em um contexto de questões de programação, focando na interação de professores com o sistema de templates e nas percepções sobre a utilidade e os desafios envolvidos. Embora não tenha sido possível aplicar o feedback automatizado aos participantes devido ao tempo limitado para o desenvolvimento deste estudo, a literatura  indica que este recurso tem potencial significativo para contribuir no aprendizado, como evidenciado por trabalhos como o de \parencite{vanpraet2024} e \parencite{fung2024} indicam que a incorporação de feedback automatizado se utilizado corretamente pode aumentar o desempenho do estudante.


\section{Metodologia do Estudo de Caso}

Este estudo adotou uma abordagem exploratória-qualitativa, utilizando como  estratégia um estudo de caso com professores de programação. O processo foi divido em cinco etapas:  

\begin{enumerate}
    \item \textbf{Apresentação da proposta}:  Foi exibido um vídeo explicativo contendo os fundamentos da geração automática de questões usando templates multicamadas, orientações de uso e exemplos práticos de aplicação. O objetivo foi garantir uma compreensão do assunto entre os participantes.
    
    \item \textbf{Criação de Templates pelos Professores}:  Os professores participantes elaboraram templates utilizando uma estrutura previamente orientada. Foi permitida uso de ferramentas de IA generativa, como o acrshort{chatgpt}, para auxiliar na criação de variações e contextos. A orientação fornecida foi divida em três topicos principais:
    \begin{itemize}
        \item \textbf{Direcionamento} : Exemplos e perguntas-chave para ajudar na formulação do enunciado e identificação dos elementos variáveis.
        \item \textbf{Questões Base} : Orientações para identificar os aspectos fundamentais do problema que desejam transformar em uma questão.
        \item \textbf{Estrutura Modelo} : Um guia de construção do template, indicando com organizar o template, variáveis e condições.
    \end{itemize}
\end{enumerate}

Os professores puderam utilziar ferramentas como o  acrshort\{chatgpt\} para explorar variações e obter sugestões adicionais. Esse suporte teve como objetivo facilitar a criação dos templates, ampliar as possibilidades de formulação e oferecer sugestões de contextos adaptáveis.

\section{Resultados Observados}

O estudo revelou que os professores conseguiram elaborar templates funcionais e diversificados, utilizando as orientações fornecidas. No entanto, alguns desafios foram identificados:

\begin{enumerate}
    \item \textbf{Dificuldade Inicial}: Professores com menor familiaridade com conceitos de questões baseada em templates demonstraram dificuldades em identificar elementos variáveis e combinar os valores.
    \item \textbf{Uso da IA Gnerativa} : A integração do ChatGPT foi bem recebida, e os participantes relataram que as sugestões fornecidas pela ferramenta ajudaram na proposta de contextos diversificados.
    \item \textbf{Percepção Geral} : Os professores consideraram o modelo útil para reduzir o tempo de elaboração de questões, no entanto preferiram que os templates já estivessem prontos pra utilizar, e após construídos pudessem acrescentar ou editar a estrutura conforme a necessidade ao invés de criar do zero.
\end{enumerate}

\section{Considerações Finais do Estudo de Caso}
A aplicação foi realizada com um número reduzido de professores, o que limita a generalização dos resultados. Estudos futuros poderão expandir o alcance para incluir mais participantes para validar a proposta. Apesar das limitações, a abordagem de geração automática de questões demonstrou um potencial significativo para criar questões em escala e reduzir o esforço necessário para a construção de conteúdos por parte dos professores. Os resultados obtidos neste estudo fornecem uma base promissora para aprimorar e expandir a abordagem em trabalhos futuros. 