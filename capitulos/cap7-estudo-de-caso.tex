\chapter{Estudo de Caso}

Esta seção apresenta o estudo de caso conduzido para avaliar a viabilidade, utilidade e relevância da proposta de geração automática de questões de programação baseada em templates multicamadas. O estudo foi estruturado para incluir a criação de templates por professores, coleta de feedback e uma análise sobre a compreensão e percepção dos participantes em relação à proposta. A seguir,  serão detalhadas as etapas do estudo e os principais resultados observados. 

\section{Objetivo do Estudo}
O objetivo principal deste estudo de caso foi validar a aplicabilidade da abordagem proposta em um contexto educacional real, focando na interação de professores com o sistema de templates e nas percepções sobre a utilidade e os desafios envolvidos. Embora não tenha sido possível aplicar o feedback automatizado aos participantes devido ao tempo limitado para o desenvolvimento deste estudo, a literatura  indica que este recurso tem potencial significativo para contribuir no aprendizado, como evidenciado por trabalhos como o de \parencite{vanpraet2024} e \parencite{fung2024}.


\section{Etapas do Estudo}

Os estudo foi dividido em cinco etapas principais : 

\begin{enumerate}
    \item \textbf{Apresentação da proposta}: Um vídeo explicativo foi elaborado para introduzir a proposta aos professores participantes. O vídeo descreveu o conceito de templates multicamadas, as etapas para criação de questões, e o uso de inteligência artificial generativa para sugerir variações e melhorias nos templates. Este material teve como objetivo assegurar o entendimento inicial e preparar os professores para a atividade prática.
    
    \item \textbf{Criação de Templates pelos Professores}: Os participantes foram convidados a criar templates de 1-layer ou n-layers, utilizando uma estrutura orientadora fornecida previamente. Essa estrutura incluiu:
    \begin{itemize}
        \item a
        \item b
        \item c
    \end{itemize}
    texto
\end{enumerate}

