% Capítulo 9
\chapter{Considerações Finais}\label{cap:ConsideracoesFinais}

Antes de iniciar a construção do modelo \LaTeX{} do \gls{ppgsc} e a escrita deste documento achava que sabia bastante sobre \LaTeX{}. Ledo engano. A quantidade de pacotes e opções para a diagramação de textos, ilustrações, referências e outros elementos é imensa e as possibilidades de configurações dos mesmos é absurda. Até mesmo em tarefas simples como a parametrização de comandos usando programação em \TeX{} e \LaTeX{}, como com \texttt{if-then-else}, possuem particularidades com as quais eu não estava familiarizado, como seu comportamento com comando expansíveis e não expansíveis. Eu apanhei muito durante esse período mas aprendi bastante.

A depuração de erros em \LaTeX{} é complicada. Às vezes um erro em um local acaba gerando uma mensagem de erro pelo processador em outro lugar distante do inicial. O uso de ferramentas de edição algumas vezes complica a depuração, visto que elas eventualmente escondem ou postergam problemas. Caso encontre um erro que não consegue eliminar ou entender porque está acontecendo, sugiro que o processe usando a linha de comando e examine qual a mensagem de erro sem o uso da ferramenta. Em alguns casos, nem isso resolve. Eu passei várias horas, distribuídas ao longo de vários dias, tentando identificar o que causava a diminuição de espaços entre os números e os títulos de capítulos, que inclusive afetava o espaçamento no Sumário! Então, decidi desabilitar vários pacotes de cada vez e examinar o resultado, até isolar o pacote que estava causando o problema.

Finalmente, faça uma busca com termos relevantes ao erro que está tentando eliminar, pois é bastante provável que alguém já teve um problema similar e o ajudaram a resolvê-lo. Por exemplo, busque ajuda na área \TeX{} do StackExchange\index{StackExchange} \url{https://tex.stackexchange.com}, que é um fórum bastante ativo que conta com muitos usuários com larga experiência em \TeX{} e \LaTeX{}.

Outro aspecto importante é o tempo de processamento. A medida que você inclui pacotes e funcionalidades, você o aumenta, é claro. Como na grande maioria dos casos, vários dos pacotes mencionados aqui não serão utilizados, eu sugiro que você desabilite o carregamento de vários pacotes e os inclua a medida que identifique que necessita deles. Outra opção é desabilitar a inclusão de capítulos nos quais você não esteja trabalhando no momento e habilitá-las posteriormente.

Finalmente, peço que sugestões de inclusões de novas funcionalidades, exemplos e de correções a este documento sejam encaminhadas para bruno@dimap.ufrn.br. Elas serão levadas em consideração na elaboração de novas versões do modelo e manual. Espero que o modelo e este documento facilitem a elaboração de sua dissertação/tese. 
