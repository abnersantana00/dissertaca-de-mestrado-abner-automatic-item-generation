% Capítulo 1
\chapter{Introdução}\label{cap:modelo}

Este capítulo apresenta uma visão geral do trabalho, destacando a contextualização e a motivação do tema, seguidas pela definição do problema e pela formulação dos objetivos (geral e específicos). Também são discutidas as questões de pesquisa que orientam o estudo, a metodologia adotada, a estrutura da dissertação e as principais contribuições e resultados esperados. 

\section{Contextualização}

A prática de resolver exercícios é amplamente reconhecida como uma das estratégias mais eficazes para melhorar o desempenho acadêmico e aumentar as chances de aprovação. Estudos focados em questões de programação demonstram que estudantes que se dedicam regularmente à resolução de problemas e à revisão dos conteúdos obtêm melhores resultados em avaliações e desenvolvem uma compreensão mais sólida dos temas abordados \parencite{Ahadi2016}. Por outro lado, aqueles que não têm o hábito de resolver exercícios frequentemente apresentam desempenho inferior, o que reforça a importância dessa prática para o aprendizado contínuo, especialmente no ensino de programação \parencite{Edwards2019}.  A criação e disponibilização de um banco de questões é fundamental neste cenário, pois amplia o acesso a exercícios de qualidade e estimula uma prática indispensável no processo de avaliação. Além disso, reduz a sobrecarga dos docentes em situações que exigem avaliações frequentes. Segundo \parencite{Puthiaparampil2020}, a ausência de um banco de questões estruturado leva os docentes a construir muitas questões  para cada avaliação, seja ela inédita ou adaptada,  demandando a esta tarefa um esforço significativo em termos de tempo e recursos, além de dificultar a consistência e a padronização das avaliações.  Um estudo realizado por \parencite{fossati2020} em cursos de introdução à programação mostrou que alunos que utilizaram banco de questões para se preparar para avaliações alcançaram uma média de 79\% de sucesso, comparado a 68\% daqueles que não utilizaram este banco de questões, demonstrando um ganho significativo de desempenho. No estudo de \parencite{Rodriguez2024}, até 82\% dos 688  estudantes que utilizaram uma plataforma com questões de programação em linguagem C atingiram um desempenho satisfatório, enquanto que quando os alunos estavam mais afastados das atividades práticas foi apresentado uma correlação próxima de zero indicando que a prática continua e regular de resolução de questões traz benefícios claros em comparação com a ausência de exercícios regulares.
Diante desse cenário, por que não construir um sistema capaz de gerar questões automaticamente e oferecer feedback personalizado em tempo real ? Técnicas de geração automática de questões têm sido cada vez mais estudadas por seu potencial de reduzir custos e escalar o processo avaliativo \parencite{kurdi2020, sewunetie2022}. Além disso, ferramentas que oferecem feedback detalhado aos alunos são fundamentais para turmas grandes, nas quais o retorno manual seria inviável \parencite{vanpraet2024, fung2024}. 
Nesse contexto, os modelos de \gls{llm}, como o \gls{gpt} tem um papel de apoio fundamental, na criação de conteúdos e no fornecimento de orientações de resposta. Essa tecnologia é amplamente utilizada na educação para oferecer dicas,  explicações detalhadas e suporte personalizado. seu usi moderado tem contribuído significativamente para a melhoria do desempenho dos alunos, conforme apontado por \parencite{yang2024}. 

\section{Problema}
Muitos alunos enfrentam dificuldades como a falta de motivação e as altas taxas de reprovação, o que evidencia a necessidade de ferramentas eficazes de praticas de exercícios e avaliação \parencite{mbiada2022}. Nos cursos introdutórios de programação, a prática de  tarefas e questionários é essencial para alcançar a proficiência em programação. Estudos sugerem que os alunos que se dedicam à prática regular de exercícios apresentam uma motivação interna maior para aprender, o que resulta em um melhor desempenho em avaliações que exigem habilidades práticas \parencite{Edwards2019}. No entanto, a criação dessas questões exige um esforço significativo por parte dos docentes, especialmente para garantir qualidade e consistência dos enunciados, tornando o processo custoso e demorado, além de algumas questões poder está suscetível a erros. 
Além disso, há um problema de escala na criação de questões, uma vez que a geração manual de um grande número de questões personalizadas é inviável para turmas numerosas. Algumas abordagens sugerem o uso de templates para geração automática de questões, como citado por \parencite{zavala2018},  mais essa estratégia enfrenta algumas limitações, pois questões geradas a partir de um único template tendem a gerar questões muito semelhantes entre si, o que pode reduzir a diversidade e o engajamento dos alunos. Outro aspecto crítico é a necessidade de fornecer feedback detalhado sobre as respostas dos alunos, visto que a ausência desse recurso pode reduzir a motivação para continuar praticando as atividades.  A proposta deste trabalho é a geração automática de questões de programação utilizando de templates multicamadas, que permitem criar uma maior diversidade de contextos nos exercícios, com o auxílio do \gls{gpt} para sugerir contextos específicos e fornecer feedback personalizado das respostas para os alunos.



\section{Justificativa}

Templates são estruturas base para criação de questões, eles funcionam como moldes com partes fixas e dinâmicas que podem ser ajustadas para gerar diferentes variações de uma mesma questão. A ideia central é que, em vez de criar questões individuais manualmente, o professor cria um template, e este é usado para gerar automaticamente várias questões a partir de um conjunto de variáveis, sendo esta uma abordagem muito comum na geração automática de questões \parencite{zavala2018}.  No entanto, criar questões usando templates em uma única camada (1-layer) apresenta limitações em termos de variabilidade, já que as manipulações são restritas a um conjunto linear de operações gerativas, por  este motivo torna as questões muito semelhantes entre si. Para superar esta limitação \parencite{lai2013} propôs uma nova abordagem baseada em templates multicamadas (n-layers), que embute um template dentro de outro, permitindo a manipulação de elementos em múltiplos níveis simultaneamente. Essa abordagem aumenta significativamente a capacidade gerativa, possibilitando a criação de questões mais ricas em contexto e com maior complexidade. Apesar de sua relevância, segundo pesquisas feitas nesta dissertação até o momento não foram identificados trabalhos acessíveis que apliquem a abordagem multicamadas especificamente na geração automática de questões de programação, destacando uma lacuna importante a ser explorada. Além disso, para este trabalho o uso de ferramentas de apoio como  \gls{gpt}, pode ser um aliado importante tanto para o professor como para o aluno, pois neste cenário pode reduzir a carga de trabalho dos docentes através da sugestão de contextos nos pontos de variações dos templates, e para o aluno que, ao responder a questão será fornecido um feedback  sobre o seu desempenho como uma forma de corrigir erros rapidamente e melhorar as soluções através do autoaprendizado.

\section{Objetivo Geral}

O objetivo geral deste trabalho é desenvolver um sistema para a geração automática de questões de programação que, a partir de templates multicamadas, seja capaz de produzir questões diversificadas e contextualizadas. Com o apoio do de ferramentas de IA como o \gls{gpt}, o sistema utilizará \textit{prompts} específicos para sugerir novos contextos às questões geradas e fornecer feedback detalhado  sobre as respostas dos alunos.

\section{Objetivos Específicos}

Os objetivos específicos deste trabalho foram definidos para orientar o desenvolvimento desta pesquisa. Eles incluem:

\begin{enumerate}[label=\textbf{\alph*)}]
    \item \textbf{Trabalhos Relacionados :} Este trabalho investiga os avanços e estudos recentes na geração automática de questões, conduzido em duas etapas principais: (i) pesquisa na literatura de palavras-chave da área de estudo e obter uma triagem inicial de títulos e resumos, e posteriormente selecionar os trabalhos mais relevantes; (ii) leitura detalhada dos textos selecionados, analisando contribuições, limitações, metodologias e alinhamento com os objetivos desta pesquisa. É importante salientar que apesar de ser uma revisão sistemática, mas com um escopo reduzido para ter a base teórica necessária ao desenvolvimento dos templates e da ferramenta proposta. 
    \item \textbf{Desenvolver os templates de múltiplas camadas :}  O segundo objetivo é construir templates de múltiplas camadas para geração automática de questões de programação. Esses templates serão projetos para incorporar diferentes elementos, como variáveis, e elementos hierárquicos, permitindo maior complexidade e abrangência das questões geradas. 
    \item \textbf{Implementar o mecanismo de combinação :} O terceiro objetivo é desenvolver um mecanismo automatizado que simplifique a complexidade na definição de combinações permitidas de valores nas questões geradas. Este mecanismo facilita o controle e a gestão das variações nos templates, permitindo a configuração de combinações dos elementos variáveis utilizados nas questões.
    \item \textbf{Realizar um Estudo de Caso :}  
O objetivo final é avaliar a proposta de utilização de templates para a geração de questões de programação, tendo como público-alvo professores de cursos introdutórios de programação. A proposta será apresentada a esses professores, e, em seguida, serão coletados dados e feedback para compreender sua relevância, viabilidade e o impacto percebido pelos professores.

 
\end{enumerate}


\section{Questões de Pesquisa}

O desenvolvimento deste trabalho é orientado por três questões de pesquisa principais, conforme apresentadas a seguir:\\

\begin{description}
    \item[\textbf{QP1}:] Quais as vantagens e desafios de criar templates para a geração automática de questões de programação, considerando limitações, diversidade e custos?
    \item[\textbf{QP2}:] Quais são os principais modelos e técnicas utilizadas na geração automática de questões, em termos de templates?
    \item[\textbf{QP3}:] É possível transformar questões existentes em templates? Quais são os benefícios e limitações dessa abordagem?
\end{description}

O próximo capítulo (\autoref{cap:metodologia})  presenta a metodologia adotada nesta pesquisa, abrangendo as etapas de revisão da literatura, o desenvolvimento da ferramenta baseada em templates multicamadas e por fim, o estudo de caso realizado com professores de programação, visando validar a relevância e a viabilidade da proposta.  




