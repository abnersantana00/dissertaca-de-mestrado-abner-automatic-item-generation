% Capítulo 1
\chapter{Introdução}\label{cap:modelo}

Este capítulo apresenta uma visão geral deste trabalho, destacando a contextualização e a motivação do tema, seguido pela definição do problema e pela formulação dos objetivos (geral e específicos). Também serão discutidas as questões de pesquisa que orientam este estudo, a metodologia adotada, a estrutura da dissertação e as principais contribuições e resultados esperados. 

\section{Contextualização}

A prática de resolver exercícios é amplamente reconhecida como uma das estratégias mais eficazes para melhorar o desempenho acadêmico e aumentar as chances de aprovação. Estudos focados em questões de programação demonstram que estudantes que se dedicam regularmente à resolução de problemas e à revisão dos conteúdos alcançam resultados excelentes em avaliações e desenvolvem uma compreensão mais aprofundada dos temas abordados \parencite{Ahadi2016}. Por outro lado,  aqueles que não possuem o hábito de resolver exercícios frequentemente apresentam um desempenho inferior, reforçando a importância de praticar exercícios, uma vez que esta prática oferece uma base sólida para prática e aprendizado contínuo,  especialmente no contexto do ensino de programação \parencite{Edwards2019}.  

A criação de um banco de questões é fundamental, pois facilita o acesso a exercícios de qualidade e estimula uma prática indispensável no processo de avaliação. Além disso, reduz a sobrecarga dos docentes em situações que exigem avaliações frequentes. . Segundo   \parencite{Puthiaparampil2020} , a ausência de um banco de questões estruturado leva os docentes a construir questões  para cada avaliação, seja ela inédita ou adaptada,  demandando a esta tarefa um esforço significativo em termos de tempo e recursos, além de dificultar a consistência e a padronização das avaliações. Então por que não construir um sistema que gere questões de forma automática e ofereça um feedback personalizado para o aluno em tempo real ? Neste cenário, surgem técnicas e ferramentas voltadas à geração automática de questões, como mencionam \parencite{kurdi2020} e \parencite{sewunetie2022}  que buscam reduzir os custos e tornar o processo de construção de questões mais rápido e escalável. Além disso, ferramentas projetadas para fornecer feedback automatizado e detalhado para exercícios de programação ganham destaque, especialmente porque, em cenários com turmas muito grandes, o feedback manual exige muitos recursos em termos de tempo, esforço e número de avaliadores \parencite{vanpraet2024} e \parencite{fung2024}.   Neste sentido, a \gls{ia} generativa, desempenha um papel fundamental, tanto na criação de conteúdos quanto no oferecimento de orientações pontuais para o estudante, esta ferramenta tem sido amplamente utilizada para oferecer orientações, dicas de resolução e explicações detalhadas, o uso moderado dessa ferramente tem contribuindo significativamente para a melhoria do desempenho dos estudantes conforme citado por \parencite{yang2024}


\section{Problema}
Muitos alunos enfrentam desafios como a falta de motivação, as altas taxas de reprovação, o que ressalta a necessidade de ferramentas de treino e avaliação eficientes \parencite{mbiada2022}. Nos cursos introdutórios de programação, a prática de  tarefas e questionários é essencial para alcançar a proficiência em programação. Estudos sugerem que os alunos que se dedicam à prática regular de exercícios tem uma motivação interna maior para aprender, o que os leva a melhor desempenho em avaliações que requerem habilidades práticas \parencite{Edwards2019}. No entanto, a criação dessas questões é um processo que exige um esforço significativo por parte dos docentes, especialmente para manter a qualidade e consistência das questões, tornando um processo custoso e demorado, e podendo está  sujeito a erros.

Além disso, existe um problema de escala na criação de questões, considerando que a geração manual de um grande número de questões personalizadas é inviável para turmas numerosas. Algumas abordagens sugerem o uso de templates para geração automática de questões como citado por \parencite{zavala2018},  mais essa estratégia enfrenta algumas limitações, uma vez que as questões geradas a partir de um template de uma camada tendem a gerar questões muito semelhantes entre si, o que pode reduzir a diversidade e o engajamento dos alunos. Outro aspecto crítico é a necessidade de fornecer feedback detalhado em tempo real sobre as respostas dos alunos, visto que a ausência desse recurso pode reduzir a motivação para continuar praticando as atividades.  O principal problema proposto neste trabalho é a geração automática de questões de programação por meio de templates multicamadas, que permitem criar uma maior diversidade de contextos nos exercícios, com o auxílio de Inteligência Artificial Generativa para gerar esses contextos específicos e fornecer feedback personalizado das respostas para os alunos em tempo real.



\section{Justificativa}

Templates são estruturas base para criação de questões, eles funcionam como modelos com partes dinâmicas que podem ser ajustadas para gerar diferentes variações de uma mesma questão. A ideia central é que, em vez de criar questões individuais manualmente, o especialista cria um template, e este é usado para gerar automaticamente várias questões a partir de um conjunto de variáveis, sendo esta uma abordagem muito comum na geração automática de questões \parencite{zavala2018}.  No entanto, criar questões usando templates em uma única camada (1-layer) apresenta limitações em termos de variabilidade, já que as manipulações são restritas a um conjunto linear de operações gerativas, por  este motivo torna as questões muito semelhantes entre si. Para superar esta limitação \parencite{lai2013} propôs uma nova abordagem baseada em templates multicamadas (n-layers), que embute um template dentro de outro, permitindo a manipulação de elementos em múltiplos níveis simultaneamente. Essa abordagem aumenta significativamente a capacidade gerativa, possibilitando a criação de questões mais ricas em contexto e com maior complexidade. Apesar de sua relevância, até o momento não foram encontrados trabalhos que apliquem essa metodologia especificamente na geração automática de questões de programação, destacando uma lacuna importante a ser explorada. Além disso, o uso de ferramentas de apoio como a IA Generativa, pode ser um aliado importante tanto para o docente como para o aluno, pois neste cenário pode reduzir a carga de trabalho dos docentes através da geração automática de contextos nos pontos de variação dos templates, e para o aluno fornecer um feedback  imediato sobre o seu desempenho como uma forma de corrigir erros rapidamente e melhorar as soluções através auto aprendizado.

\section{Objetivo Geral}

O objetivo geral deste trabalho é desenvolver um sistema inteligente para a geração automática de questões de programação que, a partir de templates multicamadas, seja capaz de produzir questões diversificadas e contextualizadas. Com o apoio da Inteligência Artificial Generativa, o sistema irá adicionar novos contextos para as questões geradas e fornecer feedback detalhado em tempo real sobre as respostas.

\section{Objetivos Específicos}

Os objetivos específicos deste trabalho foram definidos para orientar o desenvolvimento desta pesquisa. Eles incluem:

\begin{enumerate}[label=\textbf{\alph*)}]
    \item \textbf{Trabalhos Relacionados :} Este trabalho investiga os avanços e estudos recentes na geração automática de questões, conduzido em duas etapas principais: (i) pesquisa de palavras-chave com triagem inicial de títulos e resumos para selecionar os trabalhos mais relevantes; (ii) leitura detalhada dos textos selecionados, analisando contribuições, limitações, metodologias e alinhamento com os objetivos desta pesquisa. É importante salientar que não se trata de uma revisão sistemática, mas de um mapeamento de referências para fornecer a base teórica necessária ao desenvolvimento dos templates e da ferramenta proposta. 
    \item \textbf{Desenvolver os templates de múltiplas camadas :}  O segundo objetivo é construir templates de múltiplas camadas para geração automática de questões de programação. Esses templates serão projetos para incorporar diferentes elementos, como variáveis, e elementos hierárquicos, permitindo maior complexidade abrangência das questões geradas. 
    \item \textbf{Implementar o mecanismo de combinação:} O terceiro objetivo é desenvolver um mecanismo automatizado que simplifique a complexidade na definição de combinações permitidas de valores nas questões geradas. Este mecanismo facilita o controle e a gestão das variações nos templates, permitindo a configuração de combinações dos elementos variáveis utilizados nas questões.
    \item \textbf{Realizar um Estudo de Caso :}  
O objetivo final é avaliar a proposta de utilização de templates para a geração de questões de programação, tendo como público-alvo professores de cursos introdutórios de programação. A proposta será apresentada a esses docentes, e, em seguida, serão coletados dados e feedback para compreender sua relevância, viabilidade e o impacto percebido pelos professores.

 
\end{enumerate}


\section{Questões de Pesquisa}

O desenvolvimento deste trabalho é orientado por quatro questões de pesquisa principais, conforme apresentadas a seguir:\\

\begin{description}
    \item[\textbf{QP1}:] Quais as vantagens da geração automática de questões com templates comparado com a geração manual, considerando limitações, diversidade e custos?
    \item[\textbf{QP2}:] Quais são os principais modelos e técnicas utilizadas na geração automática de questões, em termos de templates?
    \item[\textbf{QP3}:] É possível transformar questões existentes em templates? Quais são os benefícios e limitações dessa abordagem?

\end{description}







