% Capítulo 3

\chapter{Trabalhos Relacionados}\label{cap:trabalhos-relacionados}

Este capítulo apresenta os principais trabalhos encontrados na literatura que possuem temas ou áreas de atuação relacionadas à proposta neste estudo. São destacados trabalhos que exploram o desenvolvimento de questões de programação baseada em templates e técnicas associadas, bem como, as palavras de busca utilizada para localizar esses trabalhos.



\section{Metodologia de Busca}

A busca pelos trabalhos relacionados foi realizada utilizando palavras-chaves definidas previamente, empregadas em bases acadêmicas reconhecidas, como IEEE Xplore, Scopus e ACM Digital Library. A expressão de busca utilizada foi: `TITLE-ABS-KEY ( ( automatic OR automated ) item AND generation AND ( programming OR exercises ) ) PUBYEAR > 2019 AND PUBYEAR < 2025'. O período considerado abrangeu de janeiro de 2018 a novembro de 2024.  O processo de busca foi estruturado para identificar pesquisas relevantes na área de geração automática de questões de programação. A busca inicial resultou em 164 artigos. Após análise preliminar baseada nos títulos e resumos, os trabalhos foram filtrados  os trabalhos com base nas questões de pesquisa, conforme os critérios de inclusão e exclusão definidos conforme apresentado a seguir. 

\subsection{Critérios de Inclusão}
\begin{itemize}
    \item Relevância direta ao tema proposto
    \item Publicação dentro do período de 2018 a 2024.
    \item Fontes confiáveis.
    \item Disponibilidade completa do texto
    \item Uso explicito de templates para geração de questões de programação.
\end{itemize}
\subsection{Critérios de Exclusão}
\begin{itemize}
    \item Trabalhos fora do escopo do uso de templates para questões de programação. 
    \item Publicações com metodologias pouco detalhadas. 
    \item O artigo não está escrito em inglês .
    \item Artigos sem acesso completo. 
\end{itemize}

\section{Processo de Seleção}
A busca resultou em 164 artigos, após a análise preliminar resultou em 10 estudos para leitura detalhada, uma triagem baseada nos critérios de inclusão e exclusão reduziu o número final para 5 estudos. Apesar de a ACM \textit{Digital Library} ter encontrado títulos e resumos altamente relevantes para este trabalho, o acesso restrito a essa base dificultou significativamente a inclusão de um maior número de estudos. A tabela a seguir apresenta a distribuição dos artigos encontrados por cada base dados e o número de estudos selecionados.

\begin{table}[htbp]
    \centering
    \begin{tabular}{|c|c|c|}
        \hline
        Base & Quantidade & Trabalhos Selecionados \\ \hline
        ACM Digital Library & 81 & 2 \\ \hline
        IEEE Xplore & 54 & 1 \\ \hline
        Scopus & 29 & 2 \\ \hline
        Total & 164 & 5 \\ \hline
    \end{tabular}
    \caption{Trabalhos Selecionados}
    \label{tab:table-trabalhos-selecionados}
\end{table}


\section{Trabalhos Selecionados}
Esta seção apresenta diferentes abordagens relacionadas à geração automática de questões de programação, com foco no uso de templates e nas limitações identificadas em termos de variação e personalização. A análise ressalta a falta de integração com tecnologias avançadas, como inteligência artificial generativa e modelos multicamadas, e explora o potencial dessas ferramentas para ampliar a diversidade das questões e oferecer feedback automatizado. Além disso, a comparação com avanços registrados em áreas como a medicina demonstra a viabilidade de adaptar essas metodologias ao contexto do ensino e avaliação em questões de programação. 

\subsection{Zavala e Mendoza 2023}

Os estudos de \parencite{zalava2018} e   \parencite{zalava2023}  apresentam contribuições complementares no campo da geração automática de questões de programação utilizando templates e dados abertos interligados (\gls{lod}). Em 2018, os autores empregaram templates com variáveis e fórmulas embutidas, resolvidas automaticamente por um programa de computador, o que permitiu a criação de grandes bancos de questões a partir de um único template. No entanto, essa metodologia apresentou limitações na diversidade, com pouca variação contextual e estrutural das questões geradas. Posteriormente, na pesquisa de \cite{zalava2023}, a incorporação de \gls{lod} possibilitou preencher automaticamente os templates com dados provenientes de fontes externas confiáveis, permitindo a criação de questões voltadas à programação e abrangendo uma gama mais ampla de tópicos. Embora essa abordagem tenha acelerado o processo de automação e reduzido erros humanos na validação de respostas, ainda enfrenta limitações, pois os templates utilizados oferecem variações predominantemente unidimensionais, sem explorar formatos mais diversificados na apresentação das questões.

 \subsection{Teubl, Ramos Batista e Zampirolli 2021}
O trabalho de \parencite{teubl2021}  apresenta o \textbf{MakeTests}, uma ferramenta que utiliza templates altamente parametrizados para gerar e corrigir automaticamente provas com diferentes estilos de questões, incluindo múltipla escolha, verdadeiro ou falso, correspondência, numérica e dissertativa. A principal contribuição da ferramenta reside na flexibilidade para criação de questões individualizadas, permitindo a geração de variações a partir de um modelo básico, o que reduz significativamente o trabalho manual dos professores.  A ferramenta também utiliza automação para correção de provas, no entanto a correção é feita por testes automatizados com respostas pré-definidas, limitando assim a interpretação das respostas dos alunos. Apesar dessas vantagens, a ferramenta requer conhecimento intermediário em Python para a criação de novos templates, o que pode ser uma barreira para alguns usuários. Portanto, há uma necessidade identificada de desenvolver uma interface mais acessível que elimine a exigência de habilidades de programação , tornando a ferramenta mais inclusiva para educadores com diferentes níveis de expertise técnica. 

\subsection{\text{Lehtinen, Santos e Sorva 2021}}
\cite {lehtinen2021} propuseram um modelo para gerar perguntas automáticas baseadas no código escrito por estudantes, conhecido como  \gls{qlcs}. O processo utiliza análise estática e dinâmica do código em conjunto com templates predefinidos para criar questões personalizadas e relevantes. Primeiramente, o código do estudante é analisado para identificar elementos padrões, como estruturas condicionais, loops, variáveis e saídas. Com base nesses elementos, o sistema seleciona templates adequados e preenche-os automaticamente com informações específicas do código analisado. Esta abordagem melhorar a compreensão dos alunos sobre seus próprios códigos. 

 \subsection{\text{Saatz 2024}}
Como proposto por Saatz (2024), a geração automática de questões pode ser usada para resolução de problemas em contextos de computação  por meio de um fluxo de trabalho estruturado em duas etapas. O autor apresenta uma abordagem baseada em modelos no primeiro estágio e no uso de \textit{templates}  no segundo para criar questões níveis de dificuldade controlada, evitando repetitividade e possíveis fraudes nas avaliações. Como principais contribuições, destaca-se a separação entre conteúdo, modelo de domínio e parâmetros de apresentação, a flexibilização na criação de diferentes tipos de questão e a simplificação do processo de criação de grandes de questões. Entre as perspectivas futuras, destaca-se a necessidade da integração com ferramentas de inteligência artificial para aprimorar personalização do feedback dos estudantes. 


\section{\text{Analise comparativa}}

Na análise comparativa apresentada na tabela \ref{tab:table-comparativa-trabalhos-selecionados}, embora todos os trabalhos selecionados utilizam templates para gerar questões, no entando, não adotam uma abordagem multicamadas nem fazem uso de IA generativa para criação de variações nas questões. Além disso, até o momento não foram encontrados estudos que utilizem templates combinados com IA generativa para sugerir variações e gerar feedback automatizado especificamente no contexto de geração de questões de programação. 

\begin{table}[htbp]
    \centering
    \begin{tabular}{|l|c|c|c|}
        \hline
        Autor& Templates & Multicamadas & IA Generativa \\ \hline
        Zavala e Mendoza 2018& \faCheck & \faClose& \faClose\\ \hline 
        Zavala e Mendoza 2023 & \faCheck&  \faClose& \faClose\\ \hline
        Teubl, Ramos Batista e Zampirolli 2021 & \faCheck& \faClose& \faClose\\\hline
 Lehtinen, Santos e SorvaSaatz 2024 & \faCheck & \faClose&\faClose \\\hline
    \end{tabular}
    \caption{Tabela Comparativa dos Trabalhos Selecionados}
    \label{tab:table-comparativa-trabalhos-selecionados}
\end{table}



Na área da saúde, como a medicina, a aplicação de templates multicamadas e de IA generativa para a geração automática de questões e feedback automatizado já é uma prática amplamente utilizada. Estudos como \parencite{falcao2023} e \parencite{kiyak2024} indicam que a IA generativa pode produzir questões de alta qualidade, contribuindo tanto para a avaliação quanto para o aprimoramento contínuo do processo de criação de questões. Esses avanços sugerem que, com as adaptações necessárias, a geração automática de questões tem um grande potencial para ser implementada com sucesso na área de programação.
No entanto, essa oportunidade ainda depende de iniciativas de pesquisadores e desenvolvedores dispostos a adaptar e aplicar as metodologias validadas na medicina ao contexto específico do desenvolvimento de questões de programação.

No próximo capítulo, serão apresentados os fundamentos teóricos que embasam este trabalho.  Serão explorados os conceitos essenciais do modelo cognitivo aplicado à elaboração de questões e o papel dos templates multicamadas na automação do processo, estabelecendo a base teórica para as discussões posteriores. 