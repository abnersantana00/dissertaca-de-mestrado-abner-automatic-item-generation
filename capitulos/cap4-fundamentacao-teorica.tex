% Capítulo 4
\chapter{Fundamentação Teórica}\label{cap:fundamentacao-teorica}

Este capítulo está organizado para introduzir os conceitos-chave e explicar detalhadamente cada seção. Serão apresentados os fundamentos teóricos que sustentam o desenvolvimento e a aplicação de dois modelos fundamentais para a implementação da proposta: o modelo cognitivo e o modelo de template multicamadas para geração automática de questões. Cada subseção abordará os conceitos essenciais relacionados a esses modelos, detalhando suas características, funcionamento e contribuições para o objetivo final do estudo.

\section{Modelo Cognitivo}
A construção de questões em larga escala, por meio de processos automatizados, tem-se tornado uma prática cada vez mais relevante na área de avaliações educacionais. Nesse contexto, a elaboração de um modelo cognitivo sólido constitui um passo fundamental para embasar a Geração Automática de Questões (\gls{aig}). De modo geral, modelos cognitivos podem ser definidos como descrições explícitas de como os estudantes processam informações e resolvem tarefas específicas, envolvendo as habilidades e os raciocínios que se espera que demonstrem em uma dada questão. O processo de construção desse modelo, conforme discutido por \parencite{gierl2021}, esta técnica consiste em identificar, organizar e documentar de forma sistematizada os conceitos, parâmetros e restrições que caracterizam tanto a criação de uma questão quanto a forma como os estudantes são esperados a resolvê-la.

A relevância do modelo cognitivo torna-se ainda mais evidente quando se busca a replicabilidade e a qualidade das questões geradas em larga escala. Esse detalhamento fornece a base para a elaboração dos templates, os quais orientam a criação de questões capazes de manter o mesmo nível de complexidade, exigência cognitiva e alinhamento ao conteúdo que se deseja avaliar. Dessa forma, o modelo cognitivo funciona como um roteiro que descreve tanto os conteúdos quanto a lógica da questão a ser gerada.

O modelo cognitivo é a base fundamental para a criação dos templates, pois organiza e descreve todos os elementos necessários para sua construção. Os templates, por sua vez, atuam como estruturas que convertem essas informações de forma efetiva, resultando em questões de avaliação individual claras e alinhadas aos objetivos de ensino.


\section{Estrutura do Modelo Cognitivo}

O modelo cognitivo funciona como um guia para organizar os elementos necessários para a construção dos templates. Estes templates servem como estruturas que traduzem as diretrizes estabelecidas no modelo em questões, claras, consistentes e alinhadas aos objetivos educacionais \parencite{keehner2017, gierl2017}. Para garantir a qualidade e replicabilidade, o modelo cognitivo deve ser estruturado de maneira a detalhar os seguintes aspectos fundamentais : 

\section{Definição de Problema e Cenários}

A primeira etapa no processo de geração de itens é o que se deseja avaliar, isto é, identificar o \emph{problema} que servirá como foco. Por exemplo, no contexto de Algoritmos, podemos focar em Estruturas de Repetição, . Assim, fica claro qual habilidade (ou conjunto de habilidades) será medida ao final.

Depois de identificar o problema, é importante detalhar o que o examinando precisa fazer ou qual tipo de conhecimento deve mobilizar. Esse passo garante que a equipe de elaboração escolha o conteúdo e o formato das questões de forma alinhada ao que efetivamente será medido.

Por fim, listam-se possíveis cenários relacionados ao problema principal, variando em formato e grau de complexidade. Em testes sobre razões e proporções, por exemplo, podemos ter cenários que envolvam a identificação direta de uma razão em um conjunto de dados, a soma de certos elementos para gerar uma proporção ou o uso de fórmulas em situações práticas. Cada cenário amplia a diversidade de itens, mas mantém a coerência com o objetivo da avaliação.

\section{Fontes de Informação (Sources of Information)}
Nesta seção, determinam-se quais conteúdos ou materiais servem de base para a elaboração dos itens. Em Matemática, por exemplo, podem-se utilizar intervalos numéricos como fonte para gerar problemas de razão ou estatística. Em outros campos, como Ciências ou Línguas, as fontes podem incluir textos, gráficos, figuras ou tabelas.

\subsection{Tipo de Fonte}
Cada fonte deve ser descrita separadamente, destacando como o examinando a utilizará ao responder a questão. Em alguns casos, uma única fonte de informação pode suprir todas as necessidades de elaboração dos itens; em outros, vários tipos de fontes precisam ser integrados (por exemplo, um enunciado textual e um gráfico).

\subsection{Organização das Fontes}
As fontes devem ser organizadas de modo sistemático, identificando-se onde e como cada elemento (dados, figuras, termos técnicos) será empregado na formulação das questões. Essa organização assegura a coerência entre o propósito de mensuração do teste e a forma pela qual o conteúdo é apresentado ao examinando.

\section{Características (Features), Elementos (Elements) e Restrições (Constraints)}
\subsection{Identificação das Características}
As \emph{Características} são as dimensões ou variáveis relevantes para a criação de novos itens. Em um problema de razão, por exemplo, tais características podem envolver “número de elementos do conjunto”, “tipo de operação aritmética” e “faixa numérica permitida”.

\subsection{Elementos (Elements) e Valores (Values)}
Cada característica contém \emph{Elementos}, que são as unidades específicas de conteúdo. No caso de um intervalo numérico de 2 a 8, cada valor inteiro (2, 3, 4, etc.) pode ser um elemento. Esses valores podem ser numéricos ou textuais, a depender do tipo de teste. A ideia é possibilitar combinações variadas de elementos que resultem em questões diversas, mas que mensurem a mesma competência.

\subsection{Restrições (Constraints)}
As restrições estabelecem as regras que evitam combinações de elementos que gerem itens inválidos ou sem sentido. Por exemplo, é possível restringir a geração de problemas que incluam números negativos quando o objetivo for mensurar apenas operações básicas com inteiros positivos. Essas regras impedem a geração aleatória e asseguram a criação de itens alinhados às metas instrucionais e ao nível de dificuldade desejado.


% ----------------------------------------------------------
% Trechos da Seção 5 (Avaliação do Modelo)
% ----------------------------------------------------------

\paragraph{}A etapa de avaliação do modelo cognitivo é fundamental para garantir que o processo de geração de itens mantenha consistência teórica e prática. Inicialmente, um especialista diferente daquele que construiu o modelo pode revisá-lo, analisando a coerência das fontes de informação, a adequação das características definidas e a pertinência das restrições estabelecidas. Esse especialista verifica se as instruções, os valores e a lógica de montagem refletem adequadamente o problema que se pretende medir, apontando eventuais inconsistências ou ambiguidades no modelo.

\paragraph{}Em seguida, recomenda-se a avaliação externa ou independente, na qual outro grupo de especialistas, que não participou do desenvolvimento prévio, fornece um parecer sobre a clareza, a completude e a precisão do modelo. O uso de escalas ou rubricas de julgamento pode sistematizar e documentar as impressões dos avaliadores, permitindo refinar ainda mais o modelo. Essa avaliação adicional contribui para validar o processo de geração de itens e para assegurar sua qualidade em aplicações futuras.




O modelo cognitivo é a base fundamental para a criação dos templates, pois organiza e descreve todos os elementos necessários para sua construção. Os templates, por sua vez, atuam como estruturas que convertem essas informações de forma efetiva, resultando em questões de avaliação individual claras e alinhadas aos objetivos de ensino.
O modelo cognitivo é o elemento base para processo da geração automática de questões, pois determina as diretrizes de como o conhecimento e as habilidades devem ser organizados para criar questões em larga escala \cite{gierl2016, gierl2017, gierlbulutzhang2018, keehner2017}. É a partir desse modelo que se definem, de maneira estruturada, aspectos como:

\begin{itemize} \item Conteúdos a serem avaliados (conceitos, regras e relações); \item Habilidades e processos de pensamento esperados dos estudantes (como identificar, analisar, resolver problemas); \item Restrições e parâmetros de combinação (limites numéricos, coerência semântica, condições mínimas para a aplicação de um conceito). \end{itemize}



\section{Relação do Modelo Cognitiva e Construção de Templates }

Por outro lado, os \textit{templates} são estruturas que convertem o modelo cognitivo em modelos práticos de questões \cite{gierl2024}. Cada \textit{template} especifica como as informações do modelo serão incorporadas ao enunciado e às alternativas (corretas e incorretas). Assim, a relação entre o modelo cognitivo e os \textit{templates} pode ser resumida nos seguintes pontos:

\begin{enumerate} \item \textbf{Definição de elementos básicos de uma questão:} O modelo cognitivo descreve quais componentes (por exemplo, dados quantitativos, termos-chave, situações-problema) precisam estar presentes para que o item seja relevante e alinhado aos objetivos pedagógicos. O \textit{template} organiza esses componentes em uma estrutura pronta para gerar diferentes versões de questão \cite{lane2016}.
\item \textbf{Combinação e manipulação de parâmetros:} Enquanto o modelo cognitivo define as regras de como os conteúdos e as habilidades podem ser combinados (por exemplo, valores ou conceitos que podem variar para alterar a dificuldade de um item), o \textit{template} coloca essas regras em prática. Em outras palavras, integra os diversos elementos para criar itens com diferentes níveis de complexidade, mantendo a coerência com a proposta original \cite{embretson2017}.

\item \textbf{Padronização e escalabilidade:} A adoção de um modelo cognitivo bem elaborado facilita a padronização do processo de elaboração de itens em grande escala. Como os \textit{templates} seguem o mesmo conjunto de regras e estruturas, as questões criadas tendem a manter consistência em termos de conteúdo, formato e nível de exigência cognitiva \cite{gierl2016, gierl2017}.

\item \textbf{Validação e ajustes contínuos:} Se um \textit{template} gerar um item que se revele muito fácil ou muito difícil, é possível identificar rapidamente se o problema está no modelo cognitivo ou no próprio \textit{template}, permitindo o ajuste da regra que originou a inconsistência. Dessa forma, a correção ocorre em nível conceitual (no modelo) ou técnico (na estrutura do \textit{template}), preservando a qualidade dos itens e mantendo-os alinhados aos propósitos da avaliação \cite{gierlbulutzhang2018}.
\end{enumerate}
Em síntese, o modelo cognitivo oferece a base teórica e estrutural, descrevendo de forma clara o que e por que está sendo avaliado, enquanto o \textit{template} representa a forma de aplicação prática dessas informações na formulação das questões \cite{gierl2024}. Essa integração torna a Geração Automática de Questões mais sólida, pois cada item criado segue a mesma lógica e mantém um padrão de qualidade, contribuindo para avaliações mais consistentes e confiáveis.

\section{Modelo de Template}

\subsubsection{Template 1-Layer}
\subsubsection{Template N-Layers}

