% Capítulo 4
\chapter{Fundamentação Teórica}\label{cap:fundamentacao-teorica}

Este capítulo está organizado para introduzir os conceitos-chave e explicar detalhadamente cada seção. Serão apresentados os fundamentos teóricos que sustentam o desenvolvimento e a aplicação de dois modelos fundamentais para a implementação da proposta: o modelo cognitivo e o modelo de template multicamadas para geração automática de questões. Cada subseção abordará os conceitos essenciais relacionados a esses modelos, detalhando suas características, funcionamento e contribuições para o objetivo final deste trabalho.

\section{Modelo Cognitivo}
A construção de questões em larga escala, por meio de processos automatizados, tem-se tornado uma prática cada vez mais relevante na área de avaliações educacionais. Nesse contexto, a elaboração de um modelo cognitivo sólido constitui um passo fundamental para embasar a Geração Automática de Questões (\gls{aig}). De modo geral, modelos cognitivos podem ser definidos como descrições explícitas de como os estudantes processam informações e resolvem tarefas específicas, envolvendo as habilidades e os raciocínios que se espera que demonstrem em uma dada questão. O processo de construção desse modelo, conforme discutido por \parencite{gierl2021}, esta técnica consiste em identificar, organizar e documentar de forma sistematizada os conceitos, parâmetros e restrições que caracterizam tanto a criação de uma questão quanto a forma como os estudantes são esperados a resolvê-la.

A relevância do modelo cognitivo fica  mais evidente quando se busca a replicabilidade e a qualidade das questões geradas. Esse detalhamento fornece a base para a elaboração dos templates, os quais orientam a criação de questões capazes de manter o mesmo nível de complexidade, exigência cognitiva e alinhamento ao conteúdo que se pretende avaliar. Dessa forma, o modelo cognitivo funciona como um roteiro que descreve tanto os conteúdos quanto a lógica da questão gerada.

O modelo cognitivo é a base fundamental para a criação dos templates, pois organiza e descreve todos os elementos necessários para sua construção. Os templates, por sua vez, atuam como estruturas que convertem essas informações de forma efetiva, resultando em questões de avaliação individual claras e alinhadas aos objetivos de ensino.


\section{Estrutura do Modelo Cognitivo}

O modelo cognitivo funciona como um guia para organizar os elementos necessários para a construção dos templates. Estes templates servem como estruturas que traduzem as diretrizes estabelecidas no modelo cognitivo em questões, claras, consistentes e alinhadas aos objetivos educacionais \parencite{keehner2017, gierl2017}. Para garantir a qualidade e replicabilidade, o modelo cognitivo deve ser estruturado de maneira a detalhar os seguintes aspectos fundamentais : 

\subsection{Problemas e Cenários}

A primeira etapa do processo de construção do modelo cognitivo consiste em definir claramente o objetivo da avaliação, identificando o problema central que servirá de base para elaboração dos cenários específicos. Em uma disciplina de algoritmos que aborda estruturas de repetição, podemos optar por explorar aspectos como o tipo de laço ou os componentes fundamentais de uma estrutura controlada por um contador. Dessa forma, podemos assegurar que as habilidades específicas a serem avaliadas estejam claramente refletidas na questão. E, posteriormente, elaborar possíveis cenários relacionados ao problema principal. Nas avaliações sobre estruturas de repetição, podemos incluir cenários que envolvam a identificação do tipo de laço utilizado em um trecho de código, a utilização de contadores para controlar iterações ou a aplicação de estruturas de repetição em situações práticas de desenvolvimento de algoritmos. Cada cenário amplia a diversidade das questões, mantendo a coerência com o objetivo da avaliação.

\subsection{Fontes de Informação}

As fontes de informação constituem o conjunto de conteúdos e matérias que podem abranger dados quantitativos, textos, fórmulas, diagramas, trechos de código, figuras ou qualquer outro elemento que possibilite aos estudantes acionar os conhecimentos e habilidades que se deseja avaliar. Cada fonte de informação deve ser descrita de forma detalhada, de modo a estabelecer claramente a relação entre o conteúdo apresentado e as características a serem avaliadas.


\subsection{Características (\textit{Features})}

As características (\textit{features}) são atributos fundamentais que compõem as dimensões ou variáveis do modelo cognitivo. Elas definem os aspectos que podem ser ajustados na criação dos templates. Na prática, cada característica é composta por três componentes principais: elementos, valores e restrições, descritos a seguir:

\begin{enumerate}
\item \textbf{Elementos} : Elementos representam as variações possíveis dentro de uma categoria ou dimensão específica da \textit{feature}, permitindo a manipulação do conteúdo através de componentes básicos que podem ser uma variável numérica, um trecho de código ou um termo técnico relevante para a questão. Os elementos devem ser construídos de forma a facilitar a combinação de um elemento com outro.

\item \textbf{Variáveis} : Valores são as possíveis combinações que os elementos podem assumir dentro de uma faixa de opções. Eles definem as variações quantitativas ou qualitativas que um elemento pode ter.

\item \textbf{Restrições} : Restrições são regras ou condições que limitam como os elementos e seus valores podem ser combinados. Elas garantem que as combinações resultantes sejam válidas, coerentes e alinhadas aos objetivos da questão, evitando a geração de questões inválidas ou sem sentido.

\end{enumerate}


A figura \ref{fig:cognitive-model} apresenta a estrutura hierárquica do modelo cognitivo para geração automática de questões, destacando os principais componentes e suas inter-relações. Ela organiza os elementos em três níveis principais : problemas e cenários, fontes de informação e características (\textit{features}). Essa visualização facilita o entendimento do processo de construção do modelo cognitivo.

\begin{figure}[ht]
	\centering
	\includegraphics[width=16cm]{./imagens/capitulo4/cognitive-model}
	\caption{Estrutura do modelo cognitivo adaptado (\cite[p. 33]{gierl2021}.)}
	\label{fig:cognitive-model}
\end{figure}


\section{Relação do Modelo Cognitivo e a Construção dos Templates }

O modelo cognitivo é o fundamento para a construção dos templates, pois organiza e descreve todos os conceitos, regras, parâmetros, limites e relações necessárias para a construção das questões. Os templates, por sua vez, convertem o modelo cognitivo em estruturas para gerar diferentes versões de questões, sendo assim a relação entre o modelo cognitivo e os templates pode ser resumida nos seguintes pontos:


\begin{enumerate} \item \textbf{Definição de elementos básicos de uma questão:} O modelo cognitivo descreve quais componentes  precisam estar presentes para que a questão seja relevante e alinhado aos objetivos de aprendizado. O \textit{template} organiza esses componentes em uma estrutura pronta para gerar diferentes versões de questão \parencite{lane2016}.

\item \textbf{Combinação e manipulação de parâmetros:}
Enquanto o modelo cognitivo define as regras de como os conteúdos e as habilidades podem ser combinados, o \textit{template} coloca essas regras em prática. Integra os diversos elementos para criar questões com diferentes níveis de complexidade, mantendo a coerência com a proposta original \parencite{embretson2017}.

\item \textbf{Padronização e escalabilidade:}
A adoção de um modelo cognitivo bem elaborado facilita a padronização do processo de elaboração de questões em grande escala. Como os \textit{templates} seguem o mesmo conjunto de regras e estruturas, as questões criadas tendem a manter consistência em termos de conteúdo, formato e nível de exigência cognitiva \parencite{gierl2017}.

\item \textbf{Validação e ajustes contínuos:}
Se um \textit{template} gerar uma questão que se revele incoerente ou sem lógica, é possível identificar rapidamente se o problema está no modelo cognitivo ou no próprio \textit{template}, permitindo o ajuste da regra que originou a inconsistência. Dessa forma, a correção ocorre em nível conceitual (no modelo) ou na estrutura do \textit{template}, preservando a qualidade das questões e mantendo-os alinhados aos propósitos da avaliação \parencite{gierlbulutzhang2018}.
\end{enumerate}


\section{Templates Multicamadas}

Modelos de templates multicamadas são estruturas utilizadas para gerar questões automaticamente com base em variações sistemáticas de componentes dentro de um template. A construção desses modelos pode seguir diferentes níveis de complexidade — como os modelos de 1 camada (\textit{1-layer}), 2 camadas (\textit{2-layers}) ou múltiplas camadas (\textit{n-layers}) — dependendo de quantas camadas de abstração e variação são incorporadas. A ideia central é permitir que, a partir de um único modelo, seja possível gerar um grande número de questões com diferentes combinações de conteúdo, mantendo a estrutura lógica do modelo . 

\subsection{Modelos de Template de uma camada (1-Layer)}

Segundo \parencite{gierl2021}, um modelo de questão pode ser definido como um template que especifica os componentes manipuláveis de uma questão, e cada componente carrega um valor ou um intervalo de valores que podem ser alterados de forma sistemática para gerar novas questões . O modelo de questão de 1-layer ("modelo de questão de camada única") constitui uma abordagem na qual se manipulam apenas alguns elementos dentro de uma estrutura fixa para produzir novas questões. Este modelo é estruturado por um único template, onde cada elemento pode assumir diferentes valores previamente estabelecido \parencite{lai2013}. A seguir será apresentado as principais características desse modelo, sua forma de aplicações e limitações.

\subsection{Estrutura do Modelo}

O modelo de 1-layer parte de um modelo cognitivo ou uma questão já existente com ponto de partida para identificar os elementos manipuláveis, e posteriormente é isolado os elementos variáveis da questão como números, termos, contextos que podem assumir diversas variações. Este modelo geralmente inclui : 

\begin{itemize}
    \item \textbf{Enunciado (Stem)} : Texto que apresenta a pergunta ou situação-problema
    \item \textbf{Elementos (\textit{Features})} : Constitui cada parte que pode mudar dentro do texto, todos os elementos estão em um só nível, compondo o corpo da questão. Esses elementos podem ser textos ou valores numéricos, em que cada elemento terá um intervalo ou conjunto de valores permitidos.
\end{itemize}

A Tabela \ref{tab:template-questoes-elementos} ilustra como a questão de referência se relaciona com o template e os respectivos elementos que podem variar. Na primeira linha, “\textbf{Questão de Referência}”, contém o enunciado-base que exemplifica o problema a ser resolvido. Na segunda linha, “\textbf{Template (\textit{Stem})}”, temos a estrutura geral do enunciado, com espaços reservados para a inserção de diferentes conteúdos possíveis, podendo ainda incluir textos fixos que complementam o enunciado.



\begin{table}[htbp]
\centering
\begin{tabular}{|l|p{10cm}|}
\hline
\textbf{Questão de Referência} 
& 
Um sistema digital classifica o tipo de acesso dos usuários com base em seu perfil. Escreva um programa que leia o tipo de usuário, utilizando \texttt{if-else}, exiba a permissão correspondente:

\begin{tabular}{|l|l|}
    \hline
    \textbf{Usuario} & \textbf{Permissão} \\
    \hline
    Visitante & Acesso Limitado \\
    Comum & Acesso Parcial \\
    Premium & Acesso Completo \\
    \hline
  \end{tabular}

Mostre na tela qual é o tipo de acesso liberado para o usuário informado.
\\
\hline

\textbf{Template (\textit{Stem})} 
& \texttt{\{contexto-geral\} \{contexto-específico\} \{tabela-condicional\} \{solicitação\}} 
\\
\hline

\textbf{Elementos (Variáveis)} 
& 
\textbf{contexto-geral:}
\begin{itemize}
  \item[{[ 0 ]}] Um sistema digital é responsável por tomar decisões automáticas com base nos dados dos usuários. 
  \item[{[ 1 ]}] Uma plataforma de atendimento inteligente analisa o perfil dos usuários para oferecer respostas personalizadas.  
\end{itemize}

\textbf{contexto-específico:}
\begin{itemize}
  \item[{[ 1 ]}] O sistema considera o tipo de usuário cadastrado para determinar permissões de acesso. 
  \item[{[ 0 ]}] A plataforma verifica a quantidade de interações feitas por semana e o tempo médio de resposta dos usuários. 
\end{itemize}

\textbf{tabela-condicional:}
\begin{itemize}
  \item[{[ 0 ]}]
  \begin{tabular}{|l|l|}
    \hline
    \textbf{Tipo de usuário} & \textbf{Permissão} \\
    \hline
    Visitante & Acesso restrito \\
    Comum & Acesso moderado \\
    Premium & Acesso total \\
    \hline
  \end{tabular}

  \item[{[ 1 ]}]
  \begin{tabular}{|l|l|}
    \hline
    \textbf{Interações semanais} & \textbf{Classificação} \\
    \hline
    $\geq 5$ & Ativo \\
    $< 5$ & Inativo \\
    \hline
  \end{tabular}
\end{itemize}

\textbf{solicitação:}
\begin{itemize}
  \item[{[ 0 ]}] Crie uma estrutura \verb|if-else| que avalie as condições apresentadas e exiba o resultado. 
  \item[{[ 1 ]}] Desenvolva um algoritmo que, com base nas variáveis, aplique as regras corretamente. 
\end{itemize}
\\
\hline
\end{tabular}
\caption{Template de questões (Elaboração Própria, 2024)}
\label{tab:template-questoes-elementos}
\end{table}




\section{Processo de geração de questões com 1-Layer}

No modelo de uma camada (\textit{1-Layer}), a geração de questões ocorre por meio da manipulação de um número limitado de elementos em um único nível. Essa abordagem é relativamente simples e de fácil implementação, pois se baseia em poucas variáveis para a criação das novas questões. Em razão dessa simplicidade, as variações ocorrem de forma linear, fazendo com que a diversidade das questões geradas seja menor. Ainda assim, este modelo é bastante útil quando se deseja alterar minimamente o contexto ou a a formulação geral, embora, por este mesmo motivo leva os estudantes a perceberem com mais facilidade padrões ou similaridades entre as questões geradas. O processo de construção de talhada é feita em três etapas principais: definição do modelo cognitivo, construção dos templates e por fim a geração das questões como será demonstrado a seguir. 

\subsection{Modelo cognitivo 1-layer}

Para elaborar um modelo cognitivo, é necessário identificar os \textbf{cenários} e \textbf{problemas} a serem tratados, as \textbf{fontes de informação} envolvidas, as \textbf{características} e \textbf{restrições} que devem ser seguidas. A partir dessa estrutura, podemos descrever a lógica que rege o comportamento do template. A Figura \ref{fig:condicional-simples}  apresenta um exemplo de modelo cognitivo de uma camada (\textit{1-layer}), baseado em problemas que envolve condicional simples. 

\begin{figure}[ht]
	\centering
	\includegraphics[width=16cm]{./imagens/capitulo4/modelo-cognitivo-1-layer}
	\caption{Modelo Congitivo : Condicional Simples  (Autoria própria, 2025)}
	\label{fig:condicional-simples}
\end{figure}

Após a construção do modelo cognitivo podemos construir um ou mais templates baseado nesta estrutura. Esses templates seguirão os mesmos princípios definidos pelo modelo, respeitando suas características, limites e restrições para garantir a coerência no desenvolvimento do template.  
O passo seguinte é a construção do template em si, tornando como base o modelo apresentado, o objetivo é transformar a lógica apresentada graficamente em estruturas reutilizáveis que podem ser aplicadas em diferentes contextos. 

\subsection{Modelo de template 1-layer}

Como ilustra a Tabela \ref{tab:template-questao}, cada elemento variável é manipulado em um único nível, modificando-se apenas os índices dos itens pontuais tais como: contexto geral, contexto específico e solicitação para a geração de novas questões. Essa estrutura mais enxuta favorece a compreensão das etapas de modelagem, sendo indicada especialmente para professores que estão em fase de aprendizado na construção de \textit{templates}. Nesse sentido, o modelo \textit{1-layer} funciona como uma base inicial, antes de se evoluir para modelos mais complexos que exigem a manipulação de múltiplos níveis, como ocorre no modelo multicamadas (\textit{n-layers}).
\begin{table}[htbp]
\centering
\begin{tabular}{|l|p{10cm}|}
\hline
\textbf{Template (Stem)} 
& \texttt{\{contexto-geral\} \{contexto-específico\} \{tabela-condicional\} \{solicitação\}} \\
\hline
\textbf{Variaveis} 
& 
\begin{minipage}[t]{\linewidth}
\vspace{0.5em}
\begin{itemize}[leftmargin=1em]
    \item {\textbf{contexto-geral}}
    \begin{itemize}
        \item [ [ 0]] asdasdsad
        \item [ [ 1]] asdaspidjhasiod
    \end{itemize}
    \item {\textbf{contexto-específico}}
    \item {\textbf{tabela-condicional}}
    \item {\textbf{solicitação}}
\end{itemize}
Escreva um programa que calcule e aplique o desconto correto, exibindo o valor economizado. 
\vspace{0.5em}
\end{minipage} \\
\hline
\end{tabular}
\caption{Template de questão - condicional simples (Autoria Própria, 2024)}
\label{tab:template-questao}
\end{table}

\subsection{Questões geradas}

asd

\section{Processo de geração de questões com  n-Layers}

A abordagem de multicamadas (\textit{n-layers}) amplia consideravelmente a capacidade de variação e a complexidade na geração de questões quando comparada ao modelo de camada única (\textit{1-layer}).  Enquanto que no modelo de uma camada é manipulado um pequeno conjunto de elementos em um \textit{template}, no modelo (\textit{n-layer}) cada camada adiciona uma nova estrutura de \textit{sub-templates} que possibilita combinar múltiplas estruturas de templates organizadas em níveis hierárquicos. Dessa forma, torna-se viável embutir os elementos, criando estruturas cada vez mais complexas que resultam em uma variedade maior  de questões geradas conforme a proposta de  \parencite{lai2013}. No entanto, essa maior diversidade traz também um aumento na complexidade de construção, pois a elaboração de modelos em múltiplas camadas requer planejamento adicional, tempo para projetar, validar e revisar cada camada e, sobretudo a experiência de quem elabora as questões \parencite{gierl2021}. 

Assim como no modelo de camada única, o ponto de partida do \textit{n-layer} é um um modelo cognitivo, que fornece os parâmetros principais de conteúdo, mas aqui cada layer pode conter blocos de informações e regras adicionais. À medida que se aumentam as camadas, introduzem-se mais cenários e variáveis e, portanto, aumentando a diversidade das questões geradas.

 CRIAR O MODELO COGNITIVO DA CALCULADORA 

Considere a questão de referência ilustrada nas Figuras \ref{fig:questao-referencia-part-1} e \ref{fig:questao-referencia-part-2}. A primeira parte descreve um menu de calculadora; a segunda detalha três categorias de operação (fácil, moderado, e difícil), baseado neste modelo. Ao migrar para uma abordagem \textit{n‑layers}, é possível substituir o contexto de “calculadora” por múltiplos cenários similares a um menu, como por exemplo, interface de computador, caixa eletrônico, jogo simples, painel de controle industrial, sem alterar estrutura e a lógica de dificuldade da questão. Se, para cada nível de dificuldade, forem especificadas 20 variações e adotarmos apenas um cenário, já obtemos 20×20×20=8.000 variações. Com dois ou três cenários, esse quantitativo pode alcançar 16.000 ou 24.000 variações respectivamente, demonstrando o poder combinatório deste modelo.


\begin{figure}[ht]
	\centering
	\includegraphics[width=12cm]{./imagens/capitulo4/questao-referencia-1.png}
	\caption{Questão de referência - parte 1 (Autoria própria, 2025)}
	\label{fig:questao-referencia-part-1}
\end{figure}


\begin{figure}[ht]
    \centering
    \includegraphics[width=12cm]{./imagens/capitulo4/questao-referencia-2.png}
    \caption{Questão de referência parte 2 - (Autoria própria, 2025)}
    \label{fig:questao-referencia-part-2}
\end{figure}

Em Paralelo com a geração automática de questões com aplicações em questões de medicina, no estudo de \parencite{gierl2021}, o template principal aborda doenças respiratórias, e as camadas seguintes eram detalhados cenários específicos para diagnosticar gripes (leve, moderada e severa) com bases nos sintomas apresentados. Aqui seguimos o mesmo raciocínio, o template principal fixa um contexto comum, e as subcamada especifica o contexto específico do contexto comum.

A Figura \ref{fig:template-1} e \ref{fig:template-2} mostra o modelo desta questão usando a abordagem multicamadas, onde o \textit{stem} (camada raiz) do template simula uma interface de uma calculadora, e os colchetes simples representam os pontos de variação, e os colchetes duplos subistituie a o subtemplate que será definido na camada seguinte, e cada indice numérico da lista é um valor possível que o template pode assumir. Isso permite reutilizar a estrutura original para substituir as variáveis sem comprometer a coerência da questão original.

\begin{figure}
    \centering
    \includegraphics[width=12cm]{./imagens/capitulo4/template-1.png}
    \caption{Template calculadora parte 1 - (Autoria própria, 2025)}
    \label{fig:template-1}
\end{figure}

\begin{figure}
    \centering
    \includegraphics[width=12cm]{./imagens/capitulo4/template-2.png}
    \caption{Template calculadora parte 2 - (Autoria própria, 2025)}
    \label{fig:template-2}
\end{figure}


\subsection{Razões para utilizar templates na geração de questões}

A geração automática de questões baseada em \textit{templates} é hoje uma estratégia versátil e operacionalmente viável para gerar questões em escala. A adoção de estruturas pré-definidas garante padronização, controle de complexidade e rastreamento dos erros, além de não exigir muitos recursos computacionais quando comparado com  as abordagens que não utilizam templates. Essas abordagens dependem de uma grande bases de dados e de modelos extensivamente treinados, que resulta em sua maioria, questões menos complexas, com criatividade limitada e com qualidade variável. Além disso, conforme \parencite{maity2024} menciona que, sistemas de geração automática de questões  baseadas em \gls{llm} tendem a gerar questões redundantes, enfrentam dificuldades na elaboração de problemas complexos, apresentam variação na qualidade das perguntas geradas e ainda estão sujeitos a riscos relacionados a vieses linguísticos. Estes fatores reforçam a atual preferencia pelo uso de \textit{templates} na geração automática de questões.


RESPONDER A QP1 AQUI, MOSTRANDO OS MODELOS E PORQUE ESOTU UTILIZANDO O MODELO DE TEMPLATES MULTICAMDAS E NÃO O MODELO DE TEMPLATES SIMPLES.