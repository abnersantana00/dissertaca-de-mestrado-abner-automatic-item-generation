\chapter{Considerações Finais}

Este trabalho demonstrou que é possível gerar muitas questões de programação usando templates multicamadas apoiados por um modelo cognitivo ou questão existente. A solução mostrou-se tecnicamente viável. Apesar do esforço inicial, cada template após construído pode entregar uma produção em escala sem custo adicional. 

\section{Resposta às Questões de Pesquisa }

\begin{description}
    \item[\textbf{QP1 - Vantagens e Desafios}:] Templates multicamadas abrangem mais casos de uso e permite variações conceituais mais complexas que templates simples. No entanto, exigem um esforço maior para serem desenvolvidos e uma revisão cuidadosa para evitar incoerências e vieses.
    \item[\textbf{QP2 - Modelos e Técnicas}:] O modelo multicamadas cria um mapeamento de habilidades e define regras de construção; o ChatGPT agiliza geração de variações e casos de teste, ampliando o escopo em relação a modelos de templates simples, mais dão trabalho pra montar.
    \item[\textbf{QP3 – Converter Questões em Templates}:] É viável extrair padrões de questões existentes e transformá-los em templates, ganhando escala rápida, consistência e reaproveitamento de material já validado; a desvantagem é o esforço manual de rotular variáveis, generalizar casos e manter o sentido original sem perder as nuances da questão original.    
\end{description}

\subsection{Limitações e Ameaças à Validade}
Apesar dos resultados encontrados, é essencial analisar cuidadosamente a validade do estudo. A seguir será destacado três fontes principais de limitação que devem ser levadas em consideração para interpretação dos dados:
\begin{itemize}
\item \textbf{Amostra restrita}: apenas oito professores participaram do estudo, o que pode limitar a generalização.
\item \textbf{Variabilidade de experiência}: diferenças no domínio técnico dos professores podem ter influenciado tanto o tempo de criação quanto a percepção de utilidade.
\item \textbf{Dependência de uma única ferramenta de IA}: os resultados podem ser diferentes com outros LLMs ou versões futuras do ChatGPT.
\end{itemize}

\section{Trabalhos Futuros}


Os trabalhos futuros foram traçados para amenizar as dificuldades relatadas pelos professores : falta de modelos-prontos, curva de aprendizagem, dificuldade em manipular JSON diretamente e depedência de ferramenta de IA. Para enfrentar esses problemas 



A continuidade deste trabalho abre espaço pra novos desenvolvimentos que visam ampliar a aplicabilidade da geração automática de questões. Neste sentido foram definidas quatro frentes que atuação:

\begin{enumerate}[label=\alph*)]
  \item \textbf{Expansão do repositório de templates}: incorporar novos conteúdos curriculares além do conteúdo fundamental, incluir tópicos avançados de programação com níveis de dificuldade, de modo a enriquecer o acervo de questões disponíveis.
  \item \textbf{Construtor visual de templates}: desenvolver uma interface gráfica \emph{no-code} que dispense a edição manual de JSON, validação automática de variáveis e pré-visualização da estrutura para professores com pouca experiência técnica em design te templates.
  \item \textbf{Avaliação em escala maior}: realizar experimentos com amostra maior de professores e estudantes, para medir impacto em ambiente real, o engajamento dos alunos e a economia de tempo dos professores.
  \item \textbf{Integração com IA generativa}: Usar a IA generativa para que a partir de questões existentes seja possível extrair estruturas de templates, sugerir automaticamente pontos de variação e gerar casos de teste.
\end{enumerate}

