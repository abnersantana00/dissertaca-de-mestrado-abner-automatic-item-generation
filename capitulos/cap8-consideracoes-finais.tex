\chapter{Considerações Finais}

Este trabalho demonstrou que é possível gerar muitas questões de programação usando templates multicamadas apoiados por um modelo cognitivo ou questão existente. A solução mostrou-se tecnicamente viável. Apesar do esforço inicial, cada template após construído pode entregar uma produção em escala sem custo adicional. 

\section{Resposta às Questões de Pesquisa }

\begin{description}
    \item[\textbf{QP1 - Vantagens e Desafios}:] Templates multicamadas abrangem mais casos de uso e permite variações conceituais mais complexas que templates simples. No entanto, exigem um esforço maior para serem desenvolvidos e uma revisão cuidadosa para evitar incoerências e vieses.
    \item[\textbf{QP2 - Modelos e Técnicas}:] O modelo multicamadas cria um mapeamento de habilidades e define regras de construção; o ChatGPT agiliza geração de variações e casos de teste, ampliando o escopo em relação a modelos de templates simples, mais dão trabalho pra montar.
    \item[\textbf{QP3 – Converter Questões em Templates}:] É viável extrair padrões de questões existentes e transformá-los em templates, ganhando escala rápida, consistência e reaproveitamento de material já validado; a desvantagem é o esforço manual de rotular variáveis, generalizar casos e manter o sentido original sem perder as nuances da questão original.    
\end{description}

\subsection{Limitações e Ameaças à Validade}
Apesar dos resultados encontrados, é essencial analisar cuidadosamente a validade do estudo. A seguir será destacado três fontes principais de limitação que devem ser levadas em consideração para interpretação dos dados:
\begin{itemize}
\item \textbf{Amostra restrita}: apenas oito professores participaram do estudo, o que pode limitar a generalização.
\item \textbf{Variabilidade de experiência}: diferenças no domínio técnico dos professores podem ter influenciado tanto o tempo de criação quanto a percepção de utilidade.
\item \textbf{Dependência de uma única ferramenta de IA}: os resultados podem ser diferentes com outros LLMs ou versões futuras do ChatGPT.
\end{itemize}

\section{Trabalhos Futuros}

Os próximos passos foram traçados para amenizar as dificuldades relatadas pelos professores : carência de modelos-prontos, curva de aprendizagem, manipular JSON diretamente e dependência de ferramentas de IA. Cada ponto abordado é baseado nos depoimentos sugerido pelos professores dos desafios e sugestões de melhoria:

\begin{enumerate}
  \item \textbf{Expansão do repositório de templates}: incorporar novos conteúdos curriculares além do conteúdo fundamental, incluir tópicos avançados de programação com níveis de dificuldade maior, de forma que o repositório reduza o tempo de aprendizado dos iniciantes e aumente a taxa de reaproveitamento do material validado.
  \item \textbf{Construtor visual de templates}: desenvolver uma interface gráfica \emph{no-code} que dispense a edição manual de JSON, e permita arrastar e soltar blocos visuais do template, e pré-visualizar os templates em tempo real com o intuito de reduzir erros de sintaxe e tornar o fluxo acessível para professores com menos experiência técnica.
  \item \textbf{Avaliação em escala maior}: conduzir um estudo com uma amostra maior comparado com ao grupo piloto de oito participantes.
  \item \textbf{Integração com IA generativa}:
  Além do ChatGPT, serão testados outros LLMs para que possa extrair estruturas de templates a partir de questões existentes, sugerir pontos de variação e gerar casos de teste automaticamente. Essa abordagem reduz a dependência de uma única ferramenta, aumenta a replicabilidade.
\end{enumerate}

Os templates multicamadas provaram ser uma boa alternativa para ampliar o conjunto de questões de programação sem perder qualidade nem a clareza das questões. Eles após serem construídos reduzem o esforço do professor, oferecem uma diversidade maior de exercícios ao aluno e mantêm a coerência pedagógica. Mesmo assim, é preciso validar essa abordagem com um público maior e uma variedade maior de contextos. É necessário também explorar outras ferramentas e abordagens de geração automática de questões para fortalecer ainda mais o ensino de programação.