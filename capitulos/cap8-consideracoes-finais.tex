\chapter{Considerações Finais}

Este trabalho demonstrou que é possível gerar muitas questões de programação usando templates multicamadas apoiados por um modelo cognitivo ou questão existente. A solução mostrou-se tecnicamente viável. Apesar do esforço inicial, cada template após construído pode entregar uma produção em escala sem custo adicional. 

\section{Resposta às Questões de Pesquisa }

\begin{description}
    \item[\textbf{QP1 - Templates Multicamadas}:] Abrangem mais casos de uso e permite variações conceituais mais complexas que templates simples. No entanto, exigem um esforço maior para serem desenvolvidos e uma revisão cuidadosa para evitar incoerências e vieses.
    \item[\textbf{QP2 - Modelos e Técnicas}:] O modelo multicamadas cria um mapeamento de habilidades e define regras de construção; o ChatGPT agiliza geração de variações e casos de teste, ampliando o escopo em relação a modelos de templates simples, mais dão trabalho pra montar.
    \item[\textbf{QP3 – Converter Questões em Templates}:] É viável extrair padrões de questões existentes e transformá-los em templates, ganhando escala rápida, consistência e reaproveitamento de material já validado; a desvantagem é o esforço manual de rotular variáveis, generalizar casos e manter o sentido original sem perder as nuances da questão original.    
\end{description}




%-----------------------------------------------------------------
\chapter{Considerações Finais}

Este trabalho investigou a geração automática de questões de programação a partir de \textbf{templates multicamadas}, fundamentada por um modelo cognitivo que assegura coerência conceitual e controle da dificuldade. Os resultados demonstraram que a abordagem é \textit{tecnicamente viável}, \textit{pedagogicamente útil} e \textit{economicamente vantajosa}: um único template pode gerar centenas de variações com custo mínimo. A conversão de questões já consolidadas em estruturas de template respondeu positivamente às três questões de pesquisa, comprovando (i) a superioridade dos modelos de templates multicamadas em relação aos de camada única, (ii) a viabilidade da transformação de questões existentes em modelos reutilizáveis, e (iii) o impacto positivo na diversidade e escalabilidade do banco de questões. Embora a elaboração inicial demande esforço e tempo — para identificar variáveis e definir restrições — o estudo de caso revelou que os professores preferem que seja disponibilizado modelos semi-prontos, indicando a necessidade de um especialista em design de templates. Entre as principais contribuições destaca-se: a formalização de modelos cognitivos orientados ao ensino de programação; definição de um formato JSON para representação de templates multicamadas; e implementação de um protótipo que integra geração de itens, execução de testes e feedback imediato mediado pela IA generativa. 

\section*{Trabalhos Futuros}

A continuidade deste trabalho abre espaço pra novos desenvolvimentos que visam ampliar a aplicabilidade da geração automática de questões. Neste sentido foram definidas quatro frentes que atuação:

\begin{enumerate}[label=\alph*)]
  \item \textbf{Expansão do repositório de templates}: incorporar novos conteúdos curriculares além do conteúdo fundamental, incluir tópicos avançados de programação com níveis de dificuldade, de modo a enriquecer o acervo de questões disponíveis.
  \item \textbf{Construtor visual de templates}: desenvolver uma interface gráfica \emph{no-code} que dispense a edição manual de JSON, validação automática de variáveis e pré-visualização da estrutura para professores com pouca experiência técnica em design te templates.
  \item \textbf{Avaliação em escala maior}: realizar experimentos com amostra maior de professores e estudantes, para medir impacto em ambiente real, o engajamento dos alunos e a economia de tempo dos professores.
  \item \textbf{Integração com IA generativa}: Usar a IA generativa para que a partir de questões existentes seja possível extrair estruturas de templates, sugerir automaticamente pontos de variação e gerar casos de teste.
\end{enumerate}


