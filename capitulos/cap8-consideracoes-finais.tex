\chapter{Considerações Finais}

Este trabalho demonstrou que é possível gerar muitas questões de programação utilizando templates multicamadas, baseados em modelos cognitivos ou em questões existentes. A solução se mostrou tecnicamente viável. Embora exista um esforço inicial na construção dos templates, uma vez prontos, eles permitem a produção em escala de questões sem custo adicional. Apesar dessa viabilidade, ainda existem desafios importantes apontados pelos professores : dificuldade para estruturar as camadas e ramificações no formato JSON, curva de aprendizado elevada, ausência de uma interface gráfica que facilite a visualização dos templates e a preferência por usar templates prontos em vez de criá-los do zero. Com base nessas evidências, apresentamos a seguir as respostas para as questões de pesquisa levantadas neste estudo.

\section{Resposta às Questões de Pesquisa }
.
\begin{itemize}
    \item \textbf{QP1 - Quais as vantagens e desafios de criar\textit{ templates} para geração automática de questões de programação, considerando limitações, diversidade e custos ?}  Criar templates para geração automática de questões traz vantagens como padronização e automatização. No entanto, há desafios com relação a limitação de variação, diversidade conceitual e custos de desenvolvimento. Templates multicamadas cobrem mais casos de teste e permitem gerar questões mais complexas, mais exigem esforço na construção e revisão, especialmente para evitar combinações inválidas.
    \item\textbf{QP2 - Quais são os modelos e técnicas utilizadas na geração automática de questões, em termos de templates ?} São utilizados modelos simples que segue uma estrutura fixa, com pouca variação de contexto. E o modelo multicamadas que permite organizar os templates em níveis hierárquicos, possibilitando criar contextos mais complexos  dentro de um mesmo template. Embora o modelo multicamadas exija mais planejamento e validação, ambos podem utilizar o ChatGPT como ferramenta de apoio, tanto na construção da estrutura quanto nos pontos de variação.
    \item\textbf{QP3 – É possível transformar questões existentes em templates ? Quais são os benefícios e limitações dessa abordagem }: Sim, e viável extrair padrões de questões existentes e transformá-los em templates. Isso permite escalar rapidamente, manter a consistência e reaproveitar materiais já validados. O principal desafio está no trabalho manual de rotular variáveis e generalizar os elementos das questões sem distorcer seu sentido original ou perder nuances importantes.    
\end{itemize}


\section{Trabalhos Futuros}

Os próximos passos foram traçados para amenizar as dificuldades relatadas pelos professores : carência de modelos-prontos, curva de aprendizagem, manipular JSON diretamente e dependência de ferramentas de IA. Cada ponto abordado é baseado nos depoimentos sugeridos pelos professores dos desafios e sugestões de melhoria:

\begin{enumerate}
  \item \textbf{Expansão do repositório de templates}: Incorporar novos conteúdos curriculares além do conteúdo fundamental, incluir tópicos avançados de programação com níveis de dificuldade maior, de forma que o repositório reduza o tempo de aprendizado dos iniciantes e aumente a taxa de reaproveitamento do material validado.
  \item \textbf{Construtor visual de templates}: Desenvolver uma interface gráfica \emph{no-code} que dispense a edição manual de JSON, e permita arrastar e soltar blocos visuais do template, e pré-visualizar os templates em tempo real com o intuito de reduzir erros de sintaxe e tornar o fluxo acessível para professores com menos experiência técnica.
  \item \textbf{Avaliação em escala maior}: Conduzir um estudo com uma amostra maior comparado com ao grupo piloto de oito participantes.
  \item \textbf{Integração com IA generativa}:
  Além do ChatGPT, serão testados outros (LLMs) para que possa extrair estruturas de templates a partir de questões existentes, sugerir pontos de variação e gerar casos de teste automaticamente. Essa abordagem visa reduz a dependência de uma única ferramenta e aumentar a replicabilidade.
\end{enumerate}

Os templates multicamadas provaram ser uma boa alternativa para ampliar o conjunto de questões de programação sem perder qualidade nem a clareza das questões. Uma vez construídos, reduzem o esforço dos professores, oferecem uma diversidade maior de exercícios para os alunos e mantêm a coerência pedagógica. Mesmo assim, é preciso validar essa abordagem com um público maior e uma variedade maior de contextos. É necessário também explorar outras ferramentas e abordagens de geração automática de questões para fortalecer ainda mais o ensino de programação.