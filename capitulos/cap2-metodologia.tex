% Capítulo 2
\chapter{Metodologia}\label{cap:metodologia}

Este estudo adota uma abordagem aplicada, com o objetivo de gerar conhecimento prático e oferecer soluções na área de avaliação educacional em programação. A pesquisa utiliza métodos quantitativos e qualitativos para a coleta e análise dos dados, conforme sugerido por \parencite{Gil2017}, que destaca a importância de abordagens abrangentes para investigações detalhadas orientadas para a prática. Como estratégia principal, foi escolhido o estudo de caso, uma vez que permite explorar situações reais com limites pouco definidos, mantendo a unidade do objeto de estudo e descrevendo seu contexto. Embora tenha sido realizada uma busca por referências teóricas, essa etapa seguiu um enfoque direcionado e seletivo, com o propósito de identificar trabalhos relevantes ao tema, sem configurar uma revisão sistemática da literatura. 

\section{Pesquisa de Trabalhos Relacionados}

A primeira etapa desta pesquisa consiste em explorar trabalhos recentes na literatura, a fim de conhecer o estado da arte e reunir a base teórica para o desenvolvimento deste estudo. Nesse processo, será realizada uma triagem com base em palavras-chave, selecionando os trabalhos por meio da análise dos títulos e resumos, seguida de uma leitura aprofundada dos trabalhos escolhidos. Essa abordagem permite identificar contribuições, desafios e lacunas relacionados à geração automática de questões de programação, fornecendo a base para responder à questão de pesquisa (\textbf{QP2}). 


\section{Desenvolvimento da Ferramenta}
Na segunda etapa, envolve a construção de um sistema capaz de executar os templates de programação com o objetivo de gerar questões diversificadas e contextualizadas. E em conjunto, será utilizado como auxilio, um mecanismo de combinação que facilita as combinações dos elementos das questões, e o uso da \gls{ia} generativa para recomendar diversos pontos de variação em cada ponto modificável do template. Após a apresentação das questão gerada e a submissão da resposta pelo aluno, a \gls{ia}  generativa fornecerá um feedback detalhado em tempo real, contribuindo para um processo de aprendizagem mais dinâmico. Estas etapas envolvem: 

\subsection{Etapas do Desenvolvimento}
Nesta seção, são descritas as principais etapas do desenvolvimento do sistema, destacando os elementos centrais que compõem o processo de criação e implementação: 

\begin{enumerate}[label=\textbf{\alph*)}]
    \item \textbf{Design dos templates :}  A criação e implementação de templates multicamadas que combinem diferentes componentes (como enunciados, variáveis, contextos e níveis de dificuldade) para geração de questões diversas, adequadas às necessidades do ensino de programação.
    \item \textbf{Uso da IA generativa} : Aplicação de técnicas de \gls{ia} generativa para auxiliar na recomendação de pontos de variação nos templates, permitindo a diversificação das questões, bem como na geração de feedback detalhado e imediato sobre as respostas dos alunos após a submissão de suas respostas. 
    \item \textbf{Implementação técnica :}  Utilização da linguagem de programação Python, utilizando bibliotecas e o framework Django. Estes recursos, por serem de código aberto e amplamente documentados foi escolhidos por serem tecnologias de software livre que oferecem alto desempenho, reuso de software e separação eficiente das camadas do sistema, conforme discutido por  \parencite{rubio2017}
    \item \textbf{Formato dos dados :}  Após comparar JSON e XML, foi escolhido o JSON para estruturar os templates, devido à sua leveza e eficiência. Essa escolha reduz o tempo de carregamento das páginas e melhora o desempenho do sistema, corroborando as evidencias apontadas por  \parencite{goyal2017} e \parencite{wang2011} .
\end{enumerate}

\section{Execução do Estudo de Caso}

\begin{enumerate}[label=\textbf{\alph*)}]
    \item \textbf{Planejamento e metodologia :}  O estudo de caso será conduzido com o objetivo de avaliar a efetividade dos templates multicamadas e do sistema de geração automática de questões. A metodologia incluirá a aplicação prática dos templates e a coleta de dados qualitativos e quantitativos para análise de desempenho, clareza e relevância das questões geradas. 

    \item \textbf{Participantes do Estudo:}  Os participantes serão professores e alunos de disciplinas introdutórias de programação.

\begin{itemize}
    \item Professores: Responsáveis por criar templates utilizando a ferramenta desenvolvida, com instruções específicas para estruturar pontos de variação nos modelos.
    \item Alunos: Resolverão questões geradas automaticamente a partir dos templates criados, abrangendo temas como entrada e saída, operadores lógicos, estruturas condicionais, laços de repetição e funções.
\end{itemize}
 
    \item \textbf{Aplicação dos Templates: }  Os professores participarão de práticas de desenvolvimento dos templates, explorando o funcionamento da ferramenta e contribuindo com novos modelos. Os alunos utilizarão os templates para resolver questões no ambiente digital, submetendo suas respostas e recebendo feedback imediato gerado pela ferramenta. 
    \item \textbf{Coleta e Análise de Dados :} coleta de dados será realizada de maneira mista:

\begin{itemize}
    \item Qualitativa: Aplicação de questionários para avaliar a clareza e relevância das questões geradas, além de entrevistas com professores para identificar desafios, esforços, limitações e benefícios percebidos durante a criação e uso dos templates.
    \item Quantitativa: Análise do desempenho dos alunos durante a resolução das questões, considerando o impacto do feedback automatizado no aprendizado.
\end{itemize}


\end{enumerate}

\section{Resultados Esperados}

1. Benefícios da Abordagem Baseada em Templates

\begin{itemize}
    \item Redução do esforço necessário para elaborar questões manualmente.
    \item Aumento da diversidade e contextualização das questões geradas.
    \item Melhoria no engajamento dos alunos, devido à disponibilização de feedback imediato e personalizado.
\end{itemize}
2. Identificação de Limitações e Oportunidades de Melhorias


Ao término do estudo, será possível responder às questões de pesquisa relativas às vantagens da geração automática de questões (QP1), aos desafios inerentes à criação de templates (QP3) e à comparação entre questões geradas automaticamente e questões manuais (QP4). Assim, o estudo contribuirá tanto para a validação da abordagem quanto para a orientação de futuros aperfeiçoamentos. 
 O estudo também buscará identificar desafios e limitações do sistema, como possíveis dificuldades na criação de templates ou limitações na variedade das questões geradas, oferecendo subsídios para aprimoramento da ferramenta.

