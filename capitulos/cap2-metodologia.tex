% Capítulo 2
\chapter{Metodologia}\label{cap:metodologia}

Este estudo adota uma abordagem aplicada, com o objetivo de gerar conhecimento prático e oferecer soluções na área de avaliação educacional em programação. A pesquisa utiliza métodos quantitativos e qualitativos para a coleta e análise dos dados, conforme sugerido por \parencite{Gil2017}, que destaca a importância de abordagens abrangentes para investigações detalhadas orientadas para a prática. Como estratégia principal, foi escolhido o estudo de caso, uma vez que permite explorar situações reais com limites pouco definidos, mantendo a unidade do objeto de estudo e descrevendo seu contexto. Embora tenha sido realizada uma busca por referências teóricas, essa etapa seguiu um enfoque direcionado e seletivo, com o propósito de identificar trabalhos relevantes ao tema, sem configurar uma revisão sistemática da literatura. 

\section{Pesquisa de Trabalhos Relacionados}

A primeira etapa desta pesquisa consiste em explorar trabalhos recentes na literatura, a fim de conhecer o estado da arte e reunir a base teórica para o desenvolvimento deste estudo. Nesse processo, será realizada uma triagem com base em palavras-chave, selecionando os trabalhos por meio da análise dos títulos e resumos, seguida de uma leitura aprofundada dos trabalhos escolhidos. Essa abordagem permite identificar contribuições, desafios e lacunas relacionados à geração automática de questões de programação, fornecendo a base para responder à questão de pesquisa \textbf{QP2}. 


\section{\textbf{ Desenvolvimento da Ferramenta} }
Na segunda etapa, envolve a construção de um sistema para executar os templates de programação, que seja capaz de gerar questões diversificadas e contextualizadas. E em conjunto, será utilizado como auxilio, um mecanismo de combinação que facilita as combinações dos elementos das questões, e o uso de \gls{ia} generativa para recomendar diversos pontos de variação em cada ponto modificável do template. Após a apresentação da questão e a resposta do aluno, a \gls{ia}  generativa fornecerá um feedback detalhado em tempo real. Estas etapas envolvem: 

(A)
(B)
(C)
(D)
E CONITNUA...
A META É : TRABALHOS RELACIONADOS, FUNDAMENTAÇÃO TEORICA E PROPOSTA CONCEITUAL (SABADO)
DOMINGO : PROPOSTA DE FERRAMENTA, ESTUDO DE CASO, CONSIDERAÇÕES FINAIS E TRABALHOS FUTUROS.