% Capítulo 2
\chapter{Metodologia}\label{cap:metodologia}

Este estudo adota uma abordagem aplicada, com o objetivo de gerar conhecimento prático e oferecer soluções na área de avaliação educacional em programação. A pesquisa utiliza métodos quantitativos e qualitativos para a coleta e análise dos dados, conforme sugerido por \parencite{Gil2017}, que destaca a importância de abordagens abrangentes para investigações detalhadas orientadas para a prática. Como estratégia principal, foi escolhido o estudo de caso, uma vez que permite explorar situações reais com limites pouco definidos, mantendo a unidade do objeto de estudo e descrevendo seu contexto. Embora tenha sido realizada uma busca por referências teóricas, essa etapa seguiu um enfoque direcionado e seletivo, com o propósito de identificar trabalhos relevantes ao tema, sem configurar uma revisão sistemática da literatura. 

\section{Pesquisa de Trabalhos Relacionados}

A primeira etapa desta pesquisa consiste em explorar trabalhos recentes na literatura, a fim de conhecer o estado da arte e reunir a base teórica para o desenvolvimento deste estudo. Nesse processo, será realizada uma triagem com base em palavras-chave, selecionando os trabalhos por meio da análise dos títulos e resumos, seguida de uma leitura aprofundada dos trabalhos escolhidos. Essa abordagem permite identificar contribuições, desafios e lacunas relacionados à geração automática de questões de programação, fornecendo a base para responder à questão de pesquisa (\textbf{QP2}). 


\section{Desenvolvimento da Ferramenta}
Na segunda etapa, envolve a construção de um sistema capaz de executar os templates de programação com o objetivo de gerar questões diversificadas e contextualizadas. E em conjunto, será utilizado como auxilio, um mecanismo de combinação que facilita as combinações dos elementos das questões, e o uso da \gls{ia} generativa para recomendar diversos pontos de variação em cada ponto modificável do template. Após a apresentação das questão gerada e a submissão da resposta pelo aluno, a \gls{ia}  generativa fornecerá um feedback detalhado em tempo real, contribuindo para um processo de aprendizagem mais dinâmico. Estas etapas envolvem: 

\subsection{Etapas do Desenvolvimento}
Nesta seção, são descritas as principais etapas do desenvolvimento do sistema, destacando os elementos centrais que compõem o processo de criação e implementação: 

\begin{enumerate}[label=\textbf{\alph*)}]
    \item \textbf{Design dos templates :}  A criação e implementação de templates multicamadas que combinem diferentes componentes (como enunciados, variáveis, contextos e níveis de dificuldade) para geração de questões diversas, adequadas às necessidades do ensino de programação.
    \item \textbf{Uso da IA generativa} : Aplicação de técnicas de \gls{ia} generativa para auxiliar na recomendação de pontos de variação nos templates, permitindo a diversificação das questões. Além disso, a IA generativa será empregada para fornecer feedback imediato e detalhado aos estudantes após a submissão de suas respostas.
    \item \textbf{Implementação técnica :}  Utilização da linguagem de programação Python, utilizando bibliotecas e o framework Django. Estes recursos, por serem de código aberto e amplamente documentados, foi escolhidos por serem tecnologias de software livre que oferecem alto desempenho, reuso de software e separação eficiente das camadas do sistema, conforme discutido por  \parencite{rubio2017}
    \item \textbf{Formato dos dados :}  Após comparar \gls{json} e \gls{xml}, foi escolhido o JSON para estruturar os templates, devido à sua leveza e eficiência. Essa escolha reduz o tempo de carregamento das páginas e melhora o desempenho do sistema, corroborando as evidencias apontadas por  \parencite{goyal2017} e \parencite{wang2011} .
\end{enumerate}

\section{Execução do Estudo de Caso}

Nesta seção, serão apresentados os principais aspectos do estudo de caso, que busca avaliar a efetividade dos templates, coletando dados qualitativos e quantitativos para analisar desempenho, clareza e relevância das questões geradas. 

\begin{enumerate}[label=\textbf{\alph*)}]
    \item \textbf{Planejamento e Metodologia:}  
    O estudo de caso será conduzido com o objetivo de avaliar a efetividade dos templates multicamadas e do sistema de geração automática de questões. A metodologia incluirá a aplicação prática dos templates e a coleta de dados qualitativos e quantitativos para análise de desempenho, dificuldade, clareza e relevância das questões geradas. 

    \item \textbf{Participantes do Estudo:}  
    Os participantes serão professores da área de computação, que lecionam disciplinas introdutórias de programação. Eles serão responsáveis por criar templates utilizando a ferramenta desenvolvida, com instruções específicas para estruturar pontos de variação nos modelos. As questões serão geradas a partir dos templates criados, abrangendo temas como entrada e saída, operadores lógicos, estruturas condicionais, laços de repetição e funções.

    \item \textbf{Aplicação dos Templates:}  
    Os professores participarão de atividades práticas de desenvolvimento de templates, explorando o funcionamento da ferramenta e contribuindo com a criação de novos modelos. As questões geradas pelos templates serão respondidas, permitindo aos participantes submeterem suas respostas e receberem feedback detalhado fornecido com o auxílio da Inteligência Artificial Generativa.

    \item \textbf{Coleta e Análise de Dados:}  A coleta de dados será realizada por meio da aplicação de questionários para avaliar a clareza e relevância das questões geradas com o intuito de identificar desafios, esforços, limitações e benefícios percebidos durante a criação e utilização dos templates. 

    


\section{Resultados Esperados}

É esperado que a abordagem de geração automática de questões baseada em templates multicamadas proporcione benefícios significativos tanto para o processo de elaboração de questões quanto para a resolução das questões. Entre esses benefícios,  destaca-se a redução do esforço necessário para criar questões tradicionais, o aumento da geração de questões em escala, questões contextualizadas e, por fim, a melhora do engajamento dos estudantes, impulsionada pela disponibilização de feedback imediato sobre suas respostas. Ao término do estudo de caso, será possível responder às questões de pesquisa relativas às vantagens da geração automática de questões (\textbf{QP1}), aos desafios inerentes à criação de templates (\textbf{QP2} e \textbf{QP3}). Assim, o estudo contribuirá para identificar limitações, oportunidades de melhorias e orientação para trabalhos futuros.