% para impressão física você pode considerar a troca de oneside para twoside
% \documentclass[a4paper, 12pt, openany, oneside, english, brazil]{scrbook} %Classe scrbook do Koma-Script

%Muda estilo da fonte usada nos títulos dos capítulos, seções e listas para a fonte usada no texto.
\addtokomafont{disposition}{\rmfamily}

%Pacote que permite evitar o erro "No room for new \write" enviando todas as escritas artravés do arquivo .aux
\usepackage{scrwfile}



%Pacotes para apêndices
\usepackage[toc,page]{appendix}


%Pacotes de linguagens, codificação de caracteres e tipografia
%Provê suporte para tipografia em várias línguas
\usepackage[brazil]{babel} % Confirme se o pacote babel-portuges está instalado.


%Pacotes de codificação e fontes
\usepackage[utf8]{inputenc} %Traduz codificações de entrada para a linguagem interna do LaTeX
%\usepackage[latin1]{inputenc}
\usepackage[T1]{fontenc}
\usepackage{fontawesome}
\usepackage{dsfont}
\usepackage{cmap}
\usepackage{lmodern}


%Pacotes de Referências
%Pacote de referências bilbiográficas
\usepackage[bibstyle=authoryear, citestyle=authoryear, maxcitenames=3, maxbibnames=20, hyperref=true, backref=true, backrefstyle=three]{biblatex} % Permite customizar citações. Mais moderno que natbib e bibtex.
%\DefineBibliographyStrings{english}{%
%	backrefpage = {page},% originally "cited on page"
%	backrefpages = {pages},% originally "cited on pages"
%}

%Pacotes de glossários e índices 

%Pacote usado para gerar índice remissivo
%Carregado antes do pacote hyperlink para que funcionem juntos
\usepackage{imakeidx}

%É preferível carregar hyperref após o biblatex, caso esteja usando-o, mas antes do glossaries!
%\usepackage[colorlinks, hyperindex, plainpages=false, pdfusetitle, pdflang=en]{hyperref}
\usepackage[colorlinks, hyperindex, plainpages=false, pdfusetitle, pdflang=pt-BR]{hyperref}

\usepackage[automake, acronym, toc]{glossaries}
%\usepackage[savewrites, nomain, acronym]{glossaries-extra} %Gerando conflito com book mas não com scrbook. Tem que instalar os módulos das linguagens a serem utilizadas. Opção nomain indica que o índice remissivo não será usado. A escolha é sua. Só tenha cuidado com o erro "No room for a new \write. \end{document}", quando o LaTeX tem muitos arquivos auxiliares abertos. A opção "savewrites" usa só um write para todos os arquivos auxiliares do glossaries, mas pode gerar problemas de referências com localizações erradas. Erro corrigido usando o pacote scrwfile



%Pacotes diversos
\usepackage{adjustbox}
\usepackage{indentfirst} % paragrafo na primeira linha escrita
\usepackage{url} %Permite quebras de linhas em certos caracteres ou combinações de caracteres
\usepackage{microtype} %Permite que se façam melhorias de justificação
\usepackage{enumitem} %Permite um maior controle sobre os layouts dos ambientes enumerate, itemize e description
\usepackage{pdflscape} %Permite que páginas sejam exibidas em formato landscape
\usepackage{hologo}%Pacote que define glifos de logos associados com TeX


%Pacotes matemáticos
\usepackage{mathtools} %Carrega automaticamente o amsmath
\usepackage{amssymb} %Pacotes da AMS (American Mathematical Society) para representação de símbolos e equações
\usepackage{esvect} %Provê estilos diferentes para setas que denotam vetores
\usepackage{siunitx} %Provê definições por nome e espaçamentos padrões para unidades padrão
\usepackage{bussproofs} %Permite criar árvores de provas no estilo de cálculo sequente
\usepackage{lplfitch} %Esses dois comandos abaixo são necessários  para que o pacote lplfitch funcione com os pacotes KOMA.
\DeclareOldFontCommand{\sf}{\normalfont\sffamily}{\mathsf}
\DeclareOldFontCommand{\bf}{\normalfont\bfseries}{\mathbf}
\usepackage{natded} %Provê comandos para mostrar provas no estilos usados por Jaśkowski e Kalish e Montague
\usepackage{zed-csp} %Suporta especificações em CSP e Z
%\usepackage{circus} %Define comandos para realizar especificações em circus


%Pacotes de objetos float e cores
%\usepackage{multirow} %Cria células que se estendem por várias linhas em ambientes tabulares 
\usepackage{graphicx} %Incluído com opção dvipdfm para eliminar erro que diz que não pode determinar tamanho de imagem. Essa opção elimina as figuras do texto!!!
\usepackage{colortbl} %Permite adicionar cores a tabelas em LaTeX
% Pacote para o uso de algoritmos
\usepackage[algochapter, linesnumbered, lined, portuguese, ruled]{algorithm2e} %Implementa o environment algorithm
\usepackage{subfig} %Permite a definição de sub-figuras
\usepackage{float} %Provê uma interface melhorada para objetos float
\usepackage{bm} %Define o comando \bm que torna seu argumento em negrito
\usepackage{setspace} %Permite ajustar facilmente espaçamento entre linhas
\usepackage{multirow} %Permite a utilização de multilinhas e multicolunas em ambientes tabulares
\usepackage[x11names]{xcolor} %Pacote que define cores de vários modelos com nomes
\usepackage[chapter]{minted} %Pacote que permite a configuração fácil de código a ser mostrado. A opção chapter realiza a numeração por capítulo
\renewcommand{\listingscaption}{Código}
\renewcommand{\listoflistingscaption}{Lista de Códigos}


%Pacotes para gerar desenhos e animações
\usepackage{qtree} %Permite desenhar diagramas de árvores
\usepackage{pgf} %Permite criar desenhos independentes de plataforma e formato, gerando saída em PS ou PDF
\usepackage{tikz} %Pacote usado para criar gráficos, figuras e ilustrações
\usepackage[tikz]{bclogo} %Permite colorir caixas virtuais ao redor de textos
\usepackage{tikz-dependency} %Provê uma bilbioteca para desenhar grafos de dependência
\usepackage{tikz-network} %Provê uma bilbioteca para desenhar redes complexas
\usepackage{tikz-3dplot} %Permite definir sistemas de coordenadas 3D para desenhar em 3D
\usetikzlibrary{switching-architectures} %Permite desenhar arquiteturas de comutação
\usetikzlibrary{mindmap} %Permite desenhar mapas mentais
\usetikzlibrary{decorations.fractals} %Permite desenhar fractais do tipo curva monstro
\usepackage{pgfplots}%Permite criar plots normais/logaritmicos em 2D/3D
\pgfplotsset{width=7cm,compat=1.17}
\pgfdeclarelayer{background} %Definição necessárias para alguns comandos TikZ
\pgfdeclarelayer{foreground} %Definição necessárias para alguns comandos TikZ
\pgfsetlayers{background, main, foreground} %some additional layers for demo
\usepackage{forest}%Permite desenhar árvores
\definecolor{folderbg}{RGB}{124,166,198}
\definecolor{folderborder}{RGB}{110,144,169}
\definecolor{bgblue}{RGB}{187,222,251}
\definecolor{bggray}{RGB}{220,220,100}

%Definições usadas para gerar desenhos de arquivos em uma ilustração de uma estrutura de árvore que não uso mais 
\def\Size{4pt}
\tikzset{
	folder/.pic={
		\filldraw[draw=folderborder,top color=folderbg!50,bottom color=folderbg]
		(-1.05*\Size,0.2\Size+5pt) rectangle ++(.75*\Size,-0.2\Size-5pt);  
		\filldraw[draw=folderborder,top color=folderbg!50,bottom color=folderbg]
		(-1.15*\Size,-\Size) rectangle (1.15*\Size,\Size);
	}
}

%Pacotes de comentários e mudanças
% Margem aumentada para receber comentários sem reclamar muito
\setlength{\marginparwidth}{2cm}
%%%%%%%%%%%%%%%%%%%%%%%%%%%%%%%%%%%%%%%%%%%%%%%%%%%%%%%%%%%%%%%%%%%
%%
%% Packages for tracking changes
%%
%%%%%%%%%%%%%%%%%%%%%%%%%%%%%%%%%%%%%%%%%%%%%%%%%%%%%%%%%%%%%%%%%%%
% Package para acompanhar mudanças e sugestões no texto
%% Use opção "final" para remover todos os comentários do texto 
% \usepackage[final]{changes}
\usepackage[draft, markup=underlined]{changes} 
\definechangesauthor[name={Bruno}, color=violet]{Bruno}
\definechangesauthor[name={Hari}, color=purple]{Hari}
\definechangesauthor[name={Salvor}, color=olive]{Salvor}

