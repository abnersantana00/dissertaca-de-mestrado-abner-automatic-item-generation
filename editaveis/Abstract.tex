% Resumo em língua estrangeira (em inglês Abstract, em espanhol Resumen, em francês Résumé)
\begin{abstract}
Developing programming skills in introductory courses requires frequent practice through exercises and assessments; however, manually creating a large number of questions is time-consuming and labor-intensive. To address this challenge, this work proposes an automatic question generation system for programming based on multi-layer templates, which enable greater variety in exercises by manipulating elements across different layers. Generative artificial intelligence serves as a supporting tool, suggesting scenarios and variations in the templates’ variable points and providing automated feedback to students after they submit their solutions. Although the initial creation of these templates demands considerable effort, once established, they streamline the process of generating new exercises. In addition, automated feedback shows potential for increasing student engagement, but further studies are needed to confirm its effectiveness and validate this approach as complementary support in the teaching-learning process. 
   
    \bigbreak

    \noindent
    \textit{Keywords}: automatic item generation, programming exercises, cognitive model, multi-layer templates, generative artificial intelligence. 
\end{abstract}