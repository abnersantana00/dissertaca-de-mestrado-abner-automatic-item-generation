% Resumo em língua vernácula
\begin{resumo}
Desenvolver habilidades de programação em cursos introdutórios exige exercícios frequentes, mas criar manualmente questões demanda muito tempo e esforço. Para enfrentar esse problema, este trabalho trata da geração automática de questões por meio de templates multicamadas, apoiados por inteligência artificial generativa para sugerir variações e fornecer feedback ao estudante. Foi desenvolvida uma ferramenta web que usa arquivos JSON para definir templates hierárquicos e instâncias de questões. A pesquisa combinou métodos quantitativos e qualitativos, com um estudo de caso com oito professores, avaliando o tempo de construção do template, a quantidade de questões geradas e as percepções por meio de questionários. A maioria dos professores produziu um template completo em 30 minutos, mas preferiu adaptar modelos prontos. As principais dificuldades encontradas foram a curva de aprendizagem, o uso direto de JSON e a falta de uma interface gráfica para visualizar as camadas. Trabalhos futuros incluem expandir o repositório de templates, criar um editor visual e avaliar a solução com amostras maiores.
  \bigbreak
  \noindent
  \textit{Palavras-chave}: geração automática de questões, exercícios de programação, modelo cognitivo, templates multicamadas, inteligência artificial generativa.
\end{resumo}