% Resumo em língua vernácula
\begin{resumo}
O desenvolvimento de habilidades de programação em cursos introdutórios requer práticas frequentes de exercícios e avaliações. No entanto, a elaboração manual de um grande volume de questões representa um desafio em termos de tempo, consistência e variedade. Neste trabalho, propõe-se um sistema de geração automática de questões de programação baseado em \textit{templates} multicamadas, ancorado em um modelo cognitivo que define regras, parâmetros e restrições para cada tópico a ser avaliado. A adoção de \textit{templates} multicamadas permite a criação de questões mais diversas e contextualizadas, pois cada camada adiciona novos elementos ou variações ao enunciado.

Além disso, a ferramenta desenvolvida utiliza IA generativa em dois aspectos fundamentais: (i) sugerir cenários e contextos nos pontos de variação dos \textit{templates} para ampliar a diversidade de exercícios; e (ii) fornecer feedback automatizado aos estudantes logo após a submissão de suas soluções, apontando erros e propondo correções. Os resultados apontam para uma redução significativa do esforço docente na elaboração de questões e para um aumento da motivação e do engajamento dos discentes, graças ao feedback imediato e à variedade de problemas gerados. Esse conjunto de técnicas contribui para a escalabilidade e a eficácia do ensino-aprendizagem em programação.

 
 
  \bigbreak

  \noindent
  \textit{Palavras-chave}: geração automática de questões, exercícios de programação, testes baseados em computador, templates multicamadas.
\end{resumo}