% Resumo em língua vernácula
\begin{resumo}
Desenvolver habilidades de programação em cursos introdutórios exige práticas frequentes de exercícios e avaliações, porém criar manualmente questões demanda muito tempo e esforço. Este trabalho propõe  gerar questões automaticamente com \textit{templates} multicamadas. O professor cria um modelo cognitivo e organiza o template em camadas hierárquicas de partes fixas e variáveis. a IA generativa auxilia em dois processos, o primeiro sugere cenários e pontos de variação para reduzir o tempo de elaboração. O segundo analisa o código enviado pelo aluno e devolve o feedback imediato. Embora a construção inicial dos templates requeira um esforço considerável, sua reutilização simplifica a criação de novos exercícios em larga escala. Estudos preliminares indicam que os professores preferem adaptar templates prontos devido a economia de tempo.  Trabalhos futuros é necessário para expandir o repertório de templates integrado com a IA generativa com uma interface gráfica simplifique a construção e aplicação em escala maior.
 
  \bigbreak

  \noindent
  \textit{Palavras-chave}: geração automática de questões, exercícios de programação, modelo cognitivo, templates multicamadas, inteligência artificial generativa.
\end{resumo}