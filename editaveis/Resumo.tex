% Resumo em língua vernácula
\begin{resumo}
Desenvolver habilidades de programação em cursos introdutórios exige práticas frequentes de exercícios e avaliações, porém criar manualmente questões demanda muito tempo e esforço. Para enfrentar esse problema, este trabalho aborda a geração automática de questões por meio de templates multicamadas, apoiados por inteligência artificial generativa para sugerir variações e fornecer feedback imediato ao estudante. Foi desenvolvido uma ferramenta web que adota arquivos JSON para definir templates hierárquicos e instâncias de questões. A pesquisa combinou métodos quantitativos e qualitativos: desenvolvendo um estudo de caso com oito professores, avaliando o tempo de construção do template, a quantidade de questões geradas e as percepções pessoais mediante a um questionário. A maioria dos professores produziram um template completo em 30 minutos, mas relataram a preferencia por adaptar templates já prontos. As principais dificuldades encontradas foram a curva de aprendizagem inicial, a manipulação direta de JSON e ausência de uma interface gráfica para visualização dos templates em camadas. Trabalhos futuros incluem expandir o repositório de templates, criar um sistema visual de manipulação de templates e avaliar a solução com amostras maiores. 
  \bigbreak

  \noindent
  \textit{Palavras-chave}: geração automática de questões, exercícios de programação, modelo cognitivo, templates multicamadas, inteligência artificial generativa.
\end{resumo}