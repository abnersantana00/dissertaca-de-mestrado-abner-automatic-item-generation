% Resumo em língua vernácula
\begin{resumo}
O desenvolvimento de habilidades de programação em cursos introdutórios exige práticas frequentes de exercícios e avaliações, porém criar manualmente um grande número de questões demanda muito tempo e esforço. Para enfrentar esse desafio, este trabalho propõe um sistema de geração automática de questões de programação baseado em \textit{templates} multicamadas, que permitem maior diversidade de exercícios ao manipular elementos em diferentes camadas. A inteligência artificial generativa atua como ferramenta de apoio, sugerindo cenários e variações nos pontos de variação  dos \textit{templates} e fornecendo feedback automatizado aos estudantes após a submissão de suas soluções. Embora a elaboração inicial dos \textit{templates} requeira um esforço considerável, uma vez estabelecidos, eles simplificam a criação de novos exercícios. Além disso, o feedback automatizado mostra potencial para aumentar o engajamento discente, mas estudos adicionais são necessários para confirmar sua eficácia e validar essa abordagem como suporte complementar ao processo de ensino-aprendizagem. 
 
  \bigbreak

  \noindent
  \textit{Palavras-chave}: geração automática de questões, exercícios de programação, modelo cognitivo, templates multicamadas, inteligência artificial generativa.
\end{resumo}