% Resumo em língua vernácula
\begin{resumo}
O desenvolvimento de habilidades de programação em cursos introdutórios exige práticas frequentes de exercícios e avaliações, porém criar manualmente um grande número de questões demanda muito tempo e esforço. Para enfrentar esse desafio, este trabalho propõe um sistema de geração automática de questões de programação baseado em templates multicamadas, que permitem maior diversidade de exercícios ao manipular elementos em diferentes camadas. A inteligência artificial gerativa atua como ferramenta de apoio complementar, sugerindo cenários e variações nos pontos de variação dos templates e fornecendo feedback automatizado aos estudantes após a submissão de suas soluções. Embora a elaboração inicial dos templates requeira um esforço considerável, seu uso simplifica a criação de novos exercícios em larga escala. Além disso, o feedback automatizado mostra potencial para aumentar o engajamento do estudante.  Além disso, o feedback automatizado mostra potencial para aumentar o engajamento discente, mas estudos adicionais são necessários para confirmar sua eficácia e validar essa abordagem como suporte complementar ao processo de ensino-aprendizagem  (falta acrescentar os resultados obtidos e ajustar o resumo ingles)
 
  \bigbreak

  \noindent
  \textit{Palavras-chave}: geração automática de questões, exercícios de programação, modelo cognitivo, templates multicamadas, inteligência artificial generativa.
\end{resumo}